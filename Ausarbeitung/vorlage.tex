%%%%%%%%%%%%%%%%%%% vorlage.tex %%%%%%%%%%%%%%%%%%%%%%%%%%%%%
%
% LaTeX-Vorlage zur Erstellung von Projekt-Dokumentationen
% im Fachbereich Informatik der Hochschule Trier
%
% Basis: Vorlage svmono des Springer Verlags
%
%%%%%%%%%%%%%%%%%%%%%%%%%%%%%%%%%%%%%%%%%%%%%%%%%%%%%%%%%%%%%

\documentclass[envcountsame,envcountchap, deutsch]{i-studis}

\usepackage{makeidx}         	% Index
\usepackage{multicol}        	% Zweispaltiger Index
%\usepackage[bottom]{footmisc}	% Erzeugung von Fu�noten

%%-----------------------------------------------------
%\newif\ifpdf
%\ifx\pdfoutput\undefined
%\pdffalse
%\else
%\pdfoutput=1
%\pdftrue
%\fi
%%--------------------------------------------------------
%\ifpdf
\usepackage[pdftex]{graphicx}
\usepackage{epstopdf}
\usepackage[pdftex,plainpages=false]{hyperref}
%\else
%\usepackage{graphicx}
%\usepackage[plainpages=false]{hyperref}
%\fi

%%-----------------------------------------------------
\usepackage{color}				% Farbverwaltung
%\usepackage{ngerman} 			% Neue deutsche Rechtsschreibung
\usepackage[english, ngerman]{babel}
\usepackage[latin1]{inputenc} 	% Erm�glicht Umlaute-Darstellung
%\usepackage[utf8]{inputenc}  	% Erm�glicht Umlaute-Darstellung unter Linux (je nach verwendetem Format)

%-----------------------------------------------------
\usepackage{listings} 			% Code-Darstellung
\lstset
{
	basicstyle=\scriptsize, 	% print whole listing small
	keywordstyle=\color{blue}\bfseries,
								% underlined bold black keywords
	identifierstyle=, 			% nothing happens
	commentstyle=\color{red}, 	% white comments
	stringstyle=\ttfamily, 		% typewriter type for strings
	showstringspaces=false, 	% no special string spaces
	framexleftmargin=7mm, 
	tabsize=3,
	showtabs=false,
	frame=single, 
	rulesepcolor=\color{blue},
	numbers=left,
	linewidth=146mm,
	xleftmargin=8mm
}
\usepackage{textcomp} 			% Celsius-Darstellung
\usepackage{amssymb,amsfonts,amstext,amsmath}	% Mathematische Symbole
\usepackage[german, ruled, vlined]{algorithm2e}
\usepackage[a4paper]{geometry} % Andere Formatierung
\usepackage{bibgerm}
\usepackage{array}
\hyphenation{Ele-men-tar-ob-jek-te  ab-ge-tas-tet Aus-wer-tung House-holder-Matrix Le-ast-Squa-res-Al-go-ri-th-men} 		% Weitere Silbentrennung bei Bedarf angeben
\setlength{\textheight}{1.1\textheight}
\pagestyle{myheadings} 			% Erzeugt selbstdefinierte Kopfzeile
\makeindex 						% Index-Erstellung


%--------------------------------------------------------------------------
\begin{document}
%------------------------- Titelblatt -------------------------------------
\title{Verbesserung und Evolution von Architekturen}
\project{Software-Architekturen Ausarbeitung}
%--------------------------------------------------------------------------
\supervisor{{Prof. Dr. Schmitz, Prof. Dr. Rock}}		% Betreuer der Arbeit
\author{Bearbeiter 1: Tobias Thierbach \\Bearbeiter 2: Erik Kierspel \\Bearbeiter 3: Annika Kremer}							% Autor der Arbeit
\groupid{Thema 8}
\address{Trier,} 							% Im Zusammenhang mit dem Datum wird hinter dem Ort ein Komma angegeben
\submitdate{17.07.2020} 				% Abgabedatum
%\begingroup
%  \renewcommand{\thepage}{title}
%  \mytitlepage
%  \newpage
%\endgroup
\begingroup
  \renewcommand{\thepage}{Titel}
  \mytitlepage
  \newpage
\endgroup
%--------------------------------------------------------------------------
\frontmatter 
%--------------------------------------------------------------------------
\kurzfassung

%% deutsch
\paragraph*{}
Im Rahmen dieser Ausarbeitung wird zun�chst vorgestellt, wie sich �nderungen anhand von Ausma�, Dimension und Zeit einordnen lassen. Danach wird die Evolutionsperspektive auf die Softwarearchitektur erl�utert. Hierbei wird zun�chst gezeigt, wie sich die Evolutionsperspektive auf verschiedenen Viewpoints, wie z.B. den Kontext, den Funktionalen, den Informations, den Concurrency und den Development  Viewpoint anwenden l�sst. Daraufhin werden die Concerns der Evolutionsperspektive erkl�rt.
Im Anschluss werden die Aktivit�ten zur Anwendung der Evolutionsperspektive anhand eines Aktivit�tsdiagramms vorgestellt. Hierbei werden die vier Schritte Anforderungen bestimmen, aktuellen Stand bestimmen, Tradeoffs abw�gen und Architekturtaktiken anwenden n�her erl�utert. AIM42 liefert dazu eine Sammlung bestehender Praktiken, mit deren Hilfe Software systematisch verbessert werden kann. Hier werden die drei Phasen von AIM42, sowie phasen�bergreifende Praktiken  beschrieben.
Nach dem allgemeinen Vorgehen wird eine Auswahl von konkreten Architectural Tactics vorgestellt, die zur Evolution von Software-Architekturen angewandt werden k�nnten. Hierbei werden die etablierten SOLID-Prinzipien im Bezug auf Evolution von Softwarearchitekturen pr�sentiert und erl�utert sowie M�glichkeiten zur lokalen Eingrenzung von �nderungsauswirkungen vorgestellt. Als weitere Taktiken wird gezeigt, wie sich Interfaces erweiterbar gestalten lassen, Variation Points einbauen und Extension Points nutzen lassen. Au�erdem werden Metamodelle behandelt, ein Architekturstil, der auf Evolution ausgelegt ist. Den Abschluss der Architectural Tactics bilden Methoden, �nderungen am Code so sicher wie m�glich umzusetzen. Zu allen Taktiken werden anschauliche Beispiele gezeigt.
Abschlie�end wird auf verschiedene Probleme und Fallen eingegangen. Dabei wird erkl�rt wie diese zustande kommen k�nnen, welche Auswirkungen sie besitzen und wie die Eintrittswahrscheinlichkeit reduzierbar ist.

%% englisch
%\paragraph*{}
%The same in english.
 			% Kurzfassung Deutsch/English
\tableofcontents 						% Inhaltsverzeichnis
%--------------------------------------------------------------------------
\mainmatter                        		% Hauptteil (ab hier arab. Seitenzahlen)
%--------------------------------------------------------------------------
% Die Kapitel werden in separaten .tex-Dateien abgelegt und hier eingebunden.
\chapter{Motivation}



\section{Einordnung von �nderungen}


\subsection{Ausma� von �nderungen}

\subsection{Dimensionen von �nderungen}

\paragraph{Funktionale Evolution}

\paragraph{Plattform Evolution}

\paragraph{Integration Evolution}

\paragraph{Wachstum}

\subsection{�nderungen �ber Zeit}



\section{Die Evolutionsperspektive}

\subsection{Warum keine Evolutionsview oder kein Evolutionsviewpoint?}

\subsection{Anwendbarkeit auf die Views}

\subsection{Concerns}
\chapter{Aktivit�ten}
In diesem Abschnitt wird das iterative Vorgehen zur Anwendung der Evolutionsperspektive beschrieben. Hierbei werden die vier Schritte Anforderungen bestimmen, Aktuellen Stand der Evolution bestimmen, Tradeoffs der Evolution abw�gen und Architekturtaktiken anwenden, erkl�rt (\cite{SSA:12}, S. 549 ff). Au�erdem wird das systematische Vorgehen des AIM42 Ansatzes, zur Evolution von Software Systemen und Architekturen erl�utert (\cite{aim42}). \\

\section{Anwendung der Evolutionsperspektive}
Schon w�hrend der Entwicklungsphase sollten alle �nderungen einer Software, die eventuell eintreffen k�nnten, erkannt und ber�cksichtigt werden. Dadurch kann sp�tere Entwicklungsarbeit f�r zum Beispiel das Hinzuf�gen neuer Funktionen, oder das Beheben von Fehlern, erheblich verk�rzt werden. Gerade bei den heutzutage verbreiteten agilen Softwareentwurfsmodellen ist die Anpassbarkeit der Software enorm wichtig. Dabei bietet sich ein iteratives Vorgehen, bestehend aus vier Schritten, zum Umgang mit diesen �nderungen an  (\cite{SSA:12}, S. 544).

\subsection{Anforderungen der Evolution bestimmen}
 Im ersten Schritt der Anwendung der Evolutionsperspektive geht es darum die Anforderungen an die Software aus der Sicht der Evolutionsperspektive zu bestimmen. Dieser Schritt ist ein weiterer Teil der Anforderungsanalyse. Hat man sich bei der initialen Anforderungsanalyse darauf konzentriert, was das System am Ende liefern muss, so geht es nun darum zu bestimmen, wie sich das System in Zukunft �ndern, beziehungsweise entwickeln k�nnte. Es geht darum, alle m�glichen Arten von Anpassungen, neuen Funktionen, neuen Schnittstellen und so weiter zu bestimmen. Diese �nderungen werden in einfacher textueller, beziehungsweise tabellarischer Form, festgehalten und jeweils nach Dimension, Ausma�, Wahrscheinlichkeit und Zeitpunkt der �nderungen eingeordnet (siehe \ref{einordnung}). \\
Die Anforderungen, die an die Evolution der Software gestellt werden, sind meist nicht explizit in den zur Verf�gung stehenden Dokumenten genannt. Aus diesem Grund muss man, um die Anforderungen zu bestimmen, die Anforderungsdokumente erneut lesen und sich darauf fokussieren, was zwischen den Zeilen steht. Hierbei kann man unter anderem nach folgenden Punkten suchen:

\begin{itemize}
	\item Aufgeschobene Funktionen, also Funktionen oder Erweiterungen, die erst zu einem sp�teren Zeitpunkt, beziehungsweise noch nicht bei Fertigstellung der Software, geliefert werden m�ssen. Im Gegensatz zu den nachfolgenden Punkten, stehen diese �nderungen explizit im Dokument.
	\item L�ckenhafte Anforderungen, sind Anforderungen an die Evolution, die bei der initialen Anforderungsanalyse nicht genauer definiert wurden, da diese unvollst�ndig ausgef�hrt wurden.
	\item Ungenaue oder undefinierte Anforderungen, sind Anforderungen die noch nicht, beziehungsweise nur ungenau definiert wurden. Diese deuten darauf hin, dass das System noch nicht richtig verstanden wurde. Befasst man sich genauer mit diesem Teil des Systems, werden dann eventuell neue Anforderungen, also �nderungen hinzukommen.
	\item Anforderungen mit offenem Ende, also Anforderungen, die implizieren, dass Funktionen oder Erweiterungen, �hnlich zu den explizit definierten Anforderungen, ben�tigt werden.
\end{itemize}

Hat man alle Dokumente nochmals untersucht und dar�ber hinaus auch mit verschiedenen Steakholdern gesprochen, k�nnen die einzelnen �nderungen zu einer Liste zusammengefasst werden. Jede dieser �nderungen ist nach Dimension, Ausma�, Wahrscheinlichkeit und Zeitpunkt der �nderungen eingeordnet. Somit stellt diese Liste die Schl�sselanforderungen der Software, aus der Evolutionsperspektive, dar. Nun geht es darum, zu verstehen, wie wichtig die �nderungen jeweils sind. Teilt man den Aufwand der jeweiligen �nderung durch die Zeitspanne, bis zu dem Zeitpunkt, an dem die �nderung erwartet wird, erh�lt man einen Wert, der die Dringlichkeit der �nderung angibt. Eine �nderung mit einem Aufwand von 20, die in 2 Monaten erwartet wird, bekommt also den Wert 10, wobei eine �nderung mit dem Aufwand von 10, die in 2 Monaten erwartet wird, lediglich 5 zugewiesen bekommt. Das hei�t, dass die erste �nderung eine h�here Priorit�t zugewiesen bekommt, als die zweite und man sich somit zun�chst um die erste �nderung k�mmern sollte. Mit diesem Prinzip lassen sich alle �nderungen nach Priorit�t sortieren und man ber�cksichtigt das Problem, dass Anforderungen sich eventuell �ber die Zeit ver�ndern. �nderungen, die erst in ferner Zukunft ben�tigt werden, bekommen eine niedrige Priorit�t zugewiesen, und man umgeht die Gefahr, dass man Arbeit in eine �nderung steckt, die vielleicht niemals eintritt. Am Ende dieses Schrittes erh�lt man eine Liste mit Schl�sselanforderungen aus der Evolutionsperspektive, die nach Dringlichkeit sortiert ist. Der Fokus in dem Projekt sollte auf den oberen Punkten der Liste liegen (\cite{SSA:12}, S. 550-551). \\

\subsection{Aktuellen Stand der Evolution bestimmen}
Im ersten Schritt wurde eine Liste mit Schl�sselanforderungen erstellt. Im zweiten geht es darum, f�r jeden Punkt dieser Liste ein Szenario zu erstellen, in dem beschrieben wird, inwiefern das System ge�ndert werden muss, um die jeweilige �nderung umzusetzen.  Es geht darum, f�r jedes dieser Szenarien zu bestimmen, wie viel dabei vom System ge�ndert werden muss und wie riskant diese �nderungen am System sind. Als Resultat erh�lt man Klarheit dar�ber, ob die Softwarearchitektur angepasst werden muss, um die neu bestimmten Anforderungen zu erf�llen. Es geht aber nicht darum jede �nderung exakt zu planen, sondern vielmehr darum, zu entscheiden, ob die jeweilige �nderung, unter Ber�cksichtigung der Kosten und des Zeitplans, �berhaupt vertretbar ist. Auch hier reicht ein textbasiertes Vorgehen aus, um die Resultate dieses Schrittes zu notieren. \\
Zusammengefasst wird in diesem Schritt zun�chst die Softwarearchitektur analysiert, um darauf basierend zu entscheiden, wie gut die �nderungen mit der zugrundeliegenden Architektur umgesetzt werden k�nnen (\cite{SSA:12} S. 551).

\subsection{Evolution Trade-offs abw�gen}
Im dritten Schritt geht es nun darum, zu untersuchen, welche Optionen f�r die Umsetzung der jeweiligen �nderungen bestehen. Hier werden nur die Punkte der Liste behandelt, die �berhaupt eine �nderung der Architektur erfordern. Um die ben�tigte Flexibilit�t, f�r die jeweiligen �nderungen zu gew�hrleisten, hat man folgende Optionen:

\begin{itemize}
	\item Ein flexibles System entwickeln. Man kann von Anfang an den Aufwand betreiben, um ein anpassbares System zu entwickeln. Dieses System ist meist modular aufgebaut und kann mit kleinerem Aufwand ge�ndert werden. Dies kostet allerdings schon in der Entwicklungsphase einen erheblichen Mehraufwand, der sich unter Umst�nden niemals auszahlt.
	\item �nderungen im Design vorsehen, aber nicht implementieren. Im Design k�nnen verschiedene �nderungen und Erweiterungen bereits vorgesehen sein, diese werden aber noch nicht implementiert, sondern auf den Zeitpunkt verschoben, an dem sie wirklich ben�tigt werden. 
	\item Keine R�cksicht auf �nderungen. W�hrend der initialen Entwicklung, ber�cksichtigt man sp�tere �nderungen nicht. Die gro�e Gefahr hierbei ist, dass kleine �nderungen, zu einem sp�teren Zeitpunkt, einen gro�en Aufwand bedeuten, beziehungsweise ein gro�es Risiko mit sich bringen.
\end{itemize}

F�r welche Option man sich entscheidet, h�ngt haupts�chlich von der Art des Systems, der Wahrscheinlichkeit f�r das Eintreten der �nderung und der Zuversicht dar�ber, �nderungen auch zu einem sp�teren Zeitpunkt ohne gro�es Risiko umsetzten zu k�nnen, ab. \\
In diesem Schritt entscheidet man sich f�r eine Strategie, mit deren Hilfe man die Anforderungen der Evolution erf�llt und zu welchem Zeitpunkt man den Aufwand in das System steckt. Dies kann bereits in der Entwicklungsphase geschehen, oder erst zu dem Zeitpunkt, an dem die �nderung ben�tigt wird. F�r jede Option muss der Einfluss auf die Architektur ber�cksichtigt werden. \\
Auch hier reicht wieder eine einfache textuelle Beschreibung der jeweiligen Strategien und den damit verbundenen Risiken und Kosten (\cite{SSA:12}, S. 552).

\subsection{Architekturtaktiken anwenden}
Im finalen Schritt wird nun die beste Strategie f�r die jeweilige �nderung angewandt. Es ist darauf zu achten, dass auch alle Views, die die Architektur der Software beschreiben, angepasst werden. \\
Da dieser Prozess iterativ ist, muss jetzt erneut der aktuelle Stand der Evolution analysiert werden und es muss entschieden werden, ob weitere �nderungen an der Architektur vonn�ten sind (\cite{SSA:12}, S. 552). Abbildung \ref{fig:activities} zeigt auf den Ablauf dieses Prozesses als Aktivit�tsdiagramm.

 \begin{figure}[htbp] 
	\centering
	\includegraphics[width=0.5\textwidth]{images/act.png}
	\caption{Prozessablauf als Aktivit�tsdiagramm (basierend auf \cite{SSA:12}, S. 549, Figure 28-1)}
	\label{fig:activities}
\end{figure}

\section{Aim42}
Je l�nger eine Software im Betrieb ist, desto riskanter, schwieriger und langwieriger werden �nderungen. Dies hat verschiedene Gr�nde, unter anderem Zeitdruck, schlechte Kommunikation und mangelnder Fokus auf langfristige Ziele im Team. Um dieses Problem zu umgehen, hat Gernot Starke, in Kooperation mit Kollegen ein Open Source Verfahren entwickelt und es \glqq Architecture Improvement Method\grqq, oder kurz aim42, genannt (\cite{improve} S. 32-33). In diesem Abschnitt wird das generelle Vorgehen erkl�rt, sowie verschiedene Praktiken und Methoden des Verfahrens beispielhaft erl�utert. 

\subsection{�berblick}
Aim42 ist eine Sammlung bereits etablierter Taktiken und Techniken zur systematischen Evolution, Wartung, Migration und Verbesserung von Software. Das iterative Vorgehen teilt sich in drei Phasen Analyze, Evaluate und Improve auf, zus�tzlich gibt es phasen�bergreifende Aktivit�ten. Der Zusammenhang wird in Abbildung \ref{fig:aim42} verdeutlicht. Durch den iterativen Ablauf von aim42 basiert das Vorgehen gr��tenteils auf dem Feedback zwischen den Phasen, dadurch ist es au�erdem gut mit der agilen Softwareentwicklung kombinierbar (\cite{aim42}).


\begin{figure}[htbp] 
	\centering
	\includegraphics[width=0.5\textwidth]{images/aim42.png}
	\caption{Die drei Phasen von aim42 (Abb. aus \cite{aim42})}
	\label{fig:aim42}
\end{figure}

\subsection{Begriffe}
Aim42 nutzt einheitliche Begriffe, die im folgenden Teilabschnitt erkl�rt werden.  Abbildung \ref{fig:aim422} zeigt dar�ber hinaus den Zusammenhang der verwendeten Begriffe.
\paragraph{Issue}
Jedes Problem, Fehler, Risiko oder suboptimale Situation, sowie deren Ursachen im System, beziehungsweise in Prozessen in Verbindung mit dem System. Dazu z�hlen unter anderem Management, Entwicklung, administrative, sowie organisatorische Aktivit�ten.
\paragraph{Cause}
Ursache f�r einen oder mehrere Fehler.
\paragraph{Improvement}
L�sung, Abhilfe oder Behebung f�r einen oder mehrere Fehler.
\paragraph{Cost (of issue)}
Kosten eines Fehlers f�r eine bestimmte Auftrittsfrequenz oder Zeitperiode.
\paragraph{Cost (of improvement)}
Kosten einer Verbesserung, Behebung, Taktik oder Strategie.
\paragraph{Risk}
Potenzielles Problem. Verbesserungen (Improvements) k�nnen Risiken (Risks), die damit in Verbindung stehen, verbessern, verschlechtern oder gar neue Risiken (Risks) hervorrufen.\\
(\cite{aim42})


%TODO quelle
\begin{figure}[htbp] 
	\centering
	\includegraphics[width=0.5\textwidth]{images/aim422.png}
	\caption{Die Terminologie von aim42 (Abb. aus \cite{aim42}) }
	\label{fig:aim422}
\end{figure}

\subsection{Analyze}
In dieser Phase geht es darum, das System zu verstehen und einen �berblick �ber den Einsatzzweck und die Qualit�tsanforderungen zu bekommen. Dieser Schritt kann als initiale Bestandsaufnahme gesehen werden. Ziel ist es interne Strukturen, Konzepte und Architekturans�tze zu erkennen, Fehler, Probleme und Workarounds zu finden und deren Ursachen, sowie eventuelle Abh�ngigkeiten zu verstehen. Dabei kann dieser Schritt als eine Art Breitensuche gesehen werden (\cite{improve} S. 34). Zu starke Konzentration auf einzelne Teilaspekte lenken eventuell von gr��eren Problemen ab. \\
Aim42 bietet f�r diese Analysephase verschiedene Praktiken an. Eine �bersicht der wichtigsten Analysepraktiken ist in Abbildung \ref{fig:analyse} zu sehen. In dieser Ausarbeitung werden zwei dieser Praktiken n�her beschrieben (\cite{aim42}).

%TODO quelle
\begin{figure}[htbp] 
	\centering
	\includegraphics[width=0.5\textwidth]{images/analyse.png}
	\caption{aim42 Analysepraktiken (Abb. aus \cite{aim42}) }
	\label{fig:analyse}
\end{figure}

\paragraph
{Context-Analysis}
Die Context-Analysis soll dazu dienen Probleme zu finden, die auf eine externe Schnittstelle zur�ckzuf�hren sind. Dabei kann es zum Beispiel sein, dass die Schnittstelle kritische Qualit�tsanforderungen beeinflusst, mit unpassenden Technologien implementiert ist oder schlecht dokumentiert, beziehungsweise schlecht verstanden ist.\\
In der Context-Analysis wird zwischen zwei Arten von Context differenziert. Zum einen der Business Context, dazu z�hlen Organisationen, Anwendungen, Benutzer oder Schnittstellen, die mit diesem System interagieren. Und zum anderen der Technical Context, dieser beschreibt Hardware oder technische Infrastruktur, die Daten oder Events zu dem System liefert. W�hrend der Business Context dazu genutzt werden kann die Gesch�ftsprozesse im Zusammenhang mit dem System zu beschreiben, dient der Technical Context dazu die Hardware Infrastruktur zu beschreiben.
Als Notation f�r die ContextAnalysis dient ein Kontextdiagramm in Verbindung mit einer Tabelle, in der die einzelnen Elemente des Kontextdiagramms beschrieben werden.\\
Abbildung \ref{fig:diag} zeigt ein solches Kontext Diagramm, das f�r das Open Source Projekt HtmlSanityCheck angefertigt wurde. In Verbindung mit der Anforderung, dass der ?check? innerhalb von 5 Sekunden fertig sein muss, ist in diesem Kontextdiagramm ersichtlich, dass die externe Schnittstelle zu anderen Webseiten zu einem Problem werden k�nnte (\cite{aim42}).


\begin{figure}[htbp] 
	\centering
	\includegraphics[width=0.5\textwidth]{images/diag.png}
	\caption{Beispiel Diagramm HtmlSanityCheck(Abb. aus \cite{aim42}) }
	\label{fig:diag}
\end{figure}

\paragraph{Capture Quality Requirements}
Diese Praktik dient dazu die spezifischen Qualit�tsanforderungen eines Systems explizit zu definieren. Als Vorgehen bietet sich hier an, alle ma�geblich beteiligten Steakholder des Projektes zu einem gemeinsamen Meeting einzuladen. Diese sollen dann w�hrend einem moderierten Workshop Qualit�tsszenarien aufschreiben, um so die spezifischen Qualit�tsanforderungen zu beschreiben. Ein Szenario beschreibt die Reaktion eines Systems auf ein bestimmtes Event. Ein Event kann zum Beispiel ein Benutzer sein, der einen Knopf anklickt, ein Administrator, der ein System neustartet oder ein Hacker, der unautorisiert Zugang zum System erlangt. Aber auch die Forderung nach einem neuen Feature, beziehungsweise ein Manager, der Operationskosten reduzieren m�chte, k�nnen ein Event sein.\\
In diesem Workshop wird oft auch nichttechnischen Steakholdern klar, dass nicht nur Business-Funktionalit�ten ben�tigt werden, sondern auch technische Anforderungen bedient werden m�ssen, um Projektziele zu erreichen.\\
Aus den vorherigen Erfahrungen hat sich ergeben, dass solche Workshops moderiert werden sollten. Au�erdem enthalten diese Szenarien oft versteckte Problembeschreibungen, Risiken und Probleme mit dem aktuellen System, diese k�nnen in die Issue List aufgenommen werden, welche am Ende des Kapitels erl�utert wird.\\
Diese Methode bietet sich an, wenn die Steakholder offen f�r Diskussion oder einen Workshop sind. Sollten alle Qualit�tsanforderungen bereits wohl definiert und auf dem aktuellen Stand sein, bietet sich diese Methode nicht an (Starke, aim42, 2020). Die Suche nach Problemen ergibt h�ufig schon sinnvolle L�sungsans�tze, oft haben die beteiligten Steakholder konkrete Vorschl�ge f�r die Verbesserung von Systemen (\cite{improve}, S. 34).

\subsection{Evaluate}
Das Management von den Verbesserungen der internen Qualit�t der Software zu �berzeugen, stellt h�ufig eine gewisse Schwierigkeit dar. Diese �nderungen sind von au�en nicht ersichtlich und bringen unter Umst�nden keine direkt erkennbare Verbesserung. Deswegen ist es wichtig Probleme und L�sungen systematisch zu vergleichen und die damit verbundenen betriebswirtschaftlichen Werte zu sch�tzen, um so verschiedene Ma�nahmen zu priorisieren (\cite{improve}, S. 34). \\\\

Ziel dieser Phase ist es also die im vorherigen Schritt gefundenen Probleme, Fehler und Risiken, vergleichbar zu machen, beziehungsweise deren Wert zu messen. Normalerweise beinhaltet diese Phase das Sch�tzen von Werten, nur in wenigen F�llen k�nnen Werte konkret gemessen werden. Auch f�r diese Phase bietet aim42 verschiedene Praktiken und Methoden an (\cite{aim42}). 

\paragraph{Estimate In Interval}
Den genauen Wert, beziehungsweise die genauen Kosten f�r Probleme, Fehler und Risiken zu sch�tzen ist oft schwierig und ungenau. Deswegen bietet sich das Sch�tzen in Intervallen an. 
Hierbei wird eine untere und eine obere Schranke f�r die jeweiligen Punkte gesch�tzt. Gute Sch�tzungen zeichnen sich dadurch aus, dass sie zu einer hohen Wahrscheinlichkeit innerhalb dieses Intervalls liegen (\cite{aim42}). Abbildung \ref{fig:cost} zeigt die Problemkosten, sowie die Kosten f�r die Ma�nahmen, welche in Intervallen gesch�tzt wurden (\cite{improve}, S. 34).


\begin{figure}[htbp] 
	\centering
	\includegraphics[width=0.9\textwidth]{images/cost.png}
	\caption{Estimate In Interval (Abb. aus \cite{improve}, S. 34, Bild 4) }
	\label{fig:cost}
\end{figure}

\subsection{Improve}
In der letzten Phase von aim42 geht es nun darum die konkreten Verbesserungen auszuf�hren und zu koordinieren, um so die Probleme zu beseitigen, die in der ersten Phase gefunden wurden. Aim42 verfolgt f�r diese Phase ein bestimmtes Konzept und liefert Fundamentals, Approches und Practices (siehe Abbildung \ref{fig:conc}), welche genutzt werden, um die Probleme zu beseitigen, beziehungsweise Verbesserungen umzusetzen. In der Softwareentwicklung werden oft mehrere strategische Planungen (Approches) in Verbindung mit verschiedenen taktischen Praktiken (Practices) �ber die Zeit genutzt (\cite{aim42}).

\begin{figure}[htbp] 
	\centering
	\includegraphics[width=0.5\textwidth]{images/conc.png}
	\caption{Improve Konzept (Abb. aus \cite{aim42}) }
	\label{fig:conc}
\end{figure}

\paragraph{Fundamentals}
\textit{Fundamentals} sind Prinzipien, die in jedem Schritt dieser Phase ber�cksichtigt werden sollten.\\
Dazu z�hlen:
\begin{itemize}
	\item \textbf{Fast Feedback } - Das Feedback sollte so schnell wie m�glich eingeholt werden, sodass eventuelle Anpassungen schnell umgesetzt werden k�nnen.
	\item \textbf{Improve Iteratively } - Alle Verbesserungen sollten inkrementell ablaufen, sodass �nderungen das System nicht negativ beeinflussen. Iterationen sind die Voraussetzung f�r das Vorgehen von aim42.
	\item \textbf{Prototype Improvement} - Die Durchf�hrbarkeit und Effektivit�t von Verbesserungen kann mit kleineren Prototypen, mit absehbarem Risiko getestet werden.
	\item \textbf{Verify After Every Change} - Nach jeder �nderung muss sichergestellt werden, dass das System einwandfrei funktioniert.
	\item \textbf{Reduce Complexity} - Einfache L�sungen sind einfacher zu verstehen und zu warten, deswegen sollte es immer das Ziel sein, die Komplexit�t m�glichst gering zu halten
	\item \textbf{Explicit Assumption} - Fehlende Fakten sollten durch explizite Annahmen kompensiert werden.
	\item \textbf{Group Improvement Actions} - �hnliche Aktivit�ten sollten gruppiert werden, sodass eventuelle Synergien genutzt werden k�nnen.
	
\end{itemize} (\cite{aim42})

\paragraph{Approaches}
Strategische Planung (Improvement Approches), sind Entscheidungen dar�ber, wie man Verbesserung generell auf lange Sicht angeht. Diese sind in die folgenden vier Kategorien aufgeteilt:

\begin{itemize}
	\item \textbf{Data Migration} - Diese Kategorie beinhaltet Ans�tze zur Datenmigration. Als generelle Vorgabe gilt, behalte die (wichtigen) Daten und verwerfe, beziehungsweise �ndere den Code.
	\item \textbf{Rewrite} - Falls Fehler am bestehenden System nicht behoben werden k�nnen und das System komplett neu entwickelt werden muss, liefert diese Kategorie Ans�tze, wie man dabei vorgehen k�nnte.
	\item \textbf{Restructure} - In dieser Kategorie geht es darum, das System zu verbessern, indem man den bestehenden Code neu strukturiert. Dabei m�ssen eventuell bestimmte Funktionen extrahiert, die Modularisierung verbessert oder fehlerhafte Stellen entfernt werden.
	\item \textbf{Improve Modularization} - Bei dieser Unterkategorie von \textit{Restructure}, geht es darum Abh�ngigkeiten, Schnittstellen und �hnliches zu verbessern
	\item \textbf{Brainsize} - Kleinere L�sungen nutzen, und gr��ere bei Bedarf aufteilen. Dies verbessert die Wartbarkeit und die Umsetzbarkeit. Hierzu z�hlen auch Microservices.
	\item \textbf{Improve Domain Focus} - Dies ist eine weitere Unterkategorie von \textit{Restructure} und beschreibt das Vorgehen zur Trennung von \glqq Dom�nenbezogenem\grqq Code und rein technischen Aspekten.
\end{itemize} (\cite{aim42})

\paragraph{Practices}
Taktische Praktiken (Practices) im Gegensatz zu den Ans�tzen (Approches) beschreiben die short-term Taktiken zur Verbesserung von Systemen. Auch die Praktiken sind in verschiedene Kategorien eingeteilt:
\begin{itemize}
	\item \textbf{Improve Processes and Organization} - Bei den Praktiken dieser Kategorie, werden Probleme verbessert, die durch ineffiziente Entwicklungs -, Rollout-, und Operation-Prozesse entstehen.
	\item \textbf{Improve Architecture and Code Structure} - Es werden alle Bereiche des Quellcodes verbessert. Dies beinhaltet Abh�ngigkeiten, Namensgebung, Style und Struktur von Code.
	\item \textbf{Improve Technical Infrastructure }- Die technische Infrastruktur beinhaltet sowohl die Verbesserung der zugrunde liegenden Software, als auch die der Hardware.
	\item \textbf{Improve Analyzability and Evaluatability}- Diese Praktiken behandeln das Erstellen, Sammeln oder Verwalten von verschiedenen Metriken oder Kennzahlen.
\end{itemize} (\cite{aim42})

\subsection{Cross-Cutting}
W�hrend des gesamten Prozesses sollten eine Problem-Liste (Issue-List), sowie die zugeh�rigen Ma�nahmen (Improvement Backlog) gef�hrt werden. Au�erdem stellt aim42 weitere phasen�bergreifende Praktiken zur Verf�gung.
\paragraph{Issue-List}
Die Issue-List ist eine Sammlung von Problemen, die in der Analysephase herausgearbeitet wurden. Jedes Problem auf der Liste, sollte zu einer, oder mehreren Ma�nahmen gelinkt sein.
Die Probleme auf der Liste sollte man vergleichbar machen, indem man ihnen ein Wert zuweist. Sie wird Phasen�bergreifend gef�hrt und bei Bedarf aktualisiert.
\paragraph{Improvement-Backlog}
Auch das Improvement-Backlog wird w�hrend des gesamten Prozesses �ffentlich gef�hrt und regelm��ig kontrolliert. Wie auch in der Issue-List, ist jede Ma�nahme auf dem Improvement-Backlog, zu einem Problem auf der Issue-List geklinkt (siehe Abbildung \ref{fig:cost}).
\\\\ (\cite{aim42})



\chapter{Studie: Einfluss der Softwarearchitektur auf die Evolution}

Da es interessant ist, den Einfluss der Softwarearchitektur, auf die Evolution, beziehungsweise die Anpassbarkeit von Software zu untersuchen, befasst sich der folgende Abschnitt mit einer Studie zu diesem Thema. \\
Der Beitrag dieses Papers ist in zweifacher Hinsicht interessant. Einmal macht es empirisch deutlich, wie wichtig es ist, Studien hinsichtlich des Einflusses der Softwarearchitektur auf die Evolution durchzuf�hren, da es weitestgehend noch nicht untersucht wurde, inwiefern �nderungen einen Einfluss auf die Softwarequalit�t besitzen. Zus�tzlich wurden zwei neue Messwerte herausgearbeitet, um den Unterschied zwischen modulinternen und modul�bergreifenden �nderungen zu quantifizieren (\cite{kour}, S. 247). Eine weitere Errungenschaft dieser Studie ist, dass sie zeigt, dass die Anwendung der empirischen Methode auf reale Daten und Open Source Softwaresysteme zeigt, dass modul�bergreifende �nderungen eher fehlerbehaftet sind, als modulinterne �nderungen (\cite{kour}, S. 247).

\section{�berblick}
Obwohl es bekannt ist, dass die Softwarearchitektur einen signifikanten Einfluss auf die Evolution hat, war es bisher schwierig, Untersuchung bez�glich eines Einflusses der Softwarearchitektur auf die Evolution durchzuf�hren, da die Softwarearchitektur in Open Source Projekten oft nicht dokumentiert ist. Die 2015 herausgebrachte Studie, auf die sich in diesem Kapitel bezogen wird, nutzt bestimmte Programme, beziehungsweise Methoden, um die Architektur von Software herauszuarbeiten. Ebenfalls versucht sie das Problem des Dokumentierens mit dem sogenannten \glqq Surrogate Architectural Views\grqq Konzept zu beheben (\cite{kour}, S. 247). Diese Herangehensweise ist demnach neu und hebt sich von vorherigen Studien ab (\cite{kour}, S. 246). \\
Ein Fokus der Studie liegt darauf, zu untersuchen, ob modul�bergreifende �nderungen einen anderen Einfluss auf die Softwarequalit�t besitzen als modulinterne �nderungen (\cite{kour}, S. 246). 
Die Studie macht ebenfalls deutlich, dass nicht alle �nderungen den gleichen Effekt haben. Das liegt daran, dass ein Modul eine limitierte Anzahl von Belangen besitzt, weshalb �nderungen, die mehrere Module besitzen, auch mehrere Belange bedienen m�ssen (\cite{kour}, S. 246). Hinzukommt, dass die Entwickler von gro�en Softwaresystemen immer nur mit einer kleinen Anzahl von Modulen vertraut sind, weshalb es f�r den Entwickler umso schwieriger ist, die Konsequenzen einer �nderung zu verstehen, umso mehr Module sie betrifft (\cite{kour}, S. 246).\\
Im Folgenden werden das methodische Vorgehen, sowie verschiedene Perspektiven der Softwarearchitektur im Mittelpunkt der Betrachtungen stehen. Au�erdem wird ein kurzer Blick auf die Ausf�hrung der Studie geworfen. \\
Drei Forschungsfragen werden diese Ausarbeitung begleiten, die innerhalb der Studie als zentral dargestellt werden. \\
Zum Schluss werden eine kurze Darstellung der Ergebnisse, eine Diskussion, sowie eine Zusammenfassung erfolgen. 

\section{Das methodische Vorgehen in der Studie}
Die Methodik der empirischen Studie setzt sich aus vier Komponenten zusammen, repr�sentiert durch die Vierecke in der folgenden Grafik (\cite{kour}, S. 248). \\

 \begin{figure}[htbp] 
	\centering
	\includegraphics[width=1\textwidth]{images/exp.png}
	\caption{Aufbau des Experiments (Abb. aus \cite{kour}, S. 248, Fig.1)}
	\label{fig:exp}
\end{figure}


Bei der ersten Komponente handelt es sich um einen �nderungsextraktor (\glqq \textit{Co-change Extractor}\grqq). Dieser sucht Quellencode repositories, und ruft die Dateigruppen auf, welche zusammen ge�ndert wurden (\cite{kour}, S. 248). Er besitzt ein modulares Design und kann erweitert werden, um auch andere repositories zu unterst�tzen (\cite{kour}, S. 248).  \\
Die zweite Komponente, genannt \glqq \textit{Defect Extractor}\grqq analysiert die \textit{commit logs} der Projekte und identifiziert die Software�nderungen, welche Fehler in das System einf�hren (\cite{kour}, S. 248). Diese beiden ersten Komponenten sind miteinander synchronisiert, um Daten zu sammeln und zu implementieren.\\
Bei der dritten Komponente handelt es sich um den \glqq \textit{Architectural Explorer}\grqq, welcher die Effekte von �nderungen auf die Ausbreitung von Fehlern innerhalb des Systems untersucht (\cite{kour}, S. 248). Sie nutzen verschiedene reverse engineering Taktiken und erstellen so einige \glqq \textit{Surrogate Views}\grqq, welche die Architektur des Systems approximieren (\cite{kour}, S. 248). \\
Die letzte Komponente ist die sogenannte \glqq\textit{ Hypothesis Testing}\grqq Komponente. Dort lassen sich die Effekte von �nderungen auf die Ausbreitung von Fehlern aus der Architekturperspektive betrachten (\cite{kour}, S. 248). \\

Innerhalb dieses Kapitels und der Studie spielen folgende drei Forschungsfragen eine elementare Rolle:
\begin{itemize}
		\item Frage 1: Sind modul�bergreifende �nderungen fehleranf�lliger als modulinterne �nderungen?
		\item Frage 2: Ergeben verschiedene Surrogates f�r die \glqq Module View\grqq unterschiedliche Ergebnisse f�r die Beziehung zwischen �nderungsverteilung und Defekten? Und falls ja, welches Surrogate liefert die besten Sch�tzungen f�r Softwarefehler?
		\item Frage 3: Hat eine Metrik, die die modul�bergreifenden �nderungen ber�cksichtigt eine h�here Korrelation mit Fehlern als eine Metrik, die die Softwarearchitektur nicht ber�cksichtigt? (\cite{kour}, S. 248)
\end{itemize}
Diese werden bei der Zusammenfassung der Ergebnisse beantwortet werden. 

\section{Die Architektur aus verschiedenen Perspektiven}
Dieses Teilkapitel befasst sich mit den Darstellungen der Architektur, die in dieser Studie genutzt wurden. Um die Architektur von komplexen Systemen zu verstehen, ist es wichtig, diese aus verschiedenen Perspektiven zu betrachten (\cite{kour}, S. 248). Im Englischen wird dies als \glqq architectural views\grqq bezeichnet. Jede davon behandelt einen anderen Belang. 
Es gibt drei verbreitete Perspektiven, die die Architektur eines Systems repr�sentieren.
Die erste wird \glqq Module View\grqq genannt, welche Einheiten der Implementation aufzeigt (\cite{kour}, S. 248).\\
Die zweite Perspektive wird als \glqq Component-and-Connector View\grqq bezeichnet. Diese Per-spektive repr�sentiert eine Menge an Modulen, die ein Laufzeitverhalten besitzen (\cite{kour}, S. 248).\\
Bei der dritten Perspektive handelt es sich um die sogenannte \glqq Allocation View\grqq, welche die Beziehung zwischen Software, Benutzer (zum Beispiel ein Team von Entwicklern) und Hardware aufzeigt (\cite{kour}, S. 248).\\
In der hier dargestellten Studie steht die \glqq Module View\grqq im Vordergrund, da sich diese Studie mit dem Aufbau und der Evolution von Software besch�ftigt und nicht mit deren Laufzeit, beziehungsweise Deployment Charakteristiken (\cite{kour}, S. 249).

\section{Ausf�hrung der Analyse}
Die folgende Darstellung wird sich damit besch�ftigen, wie es m�glich war, die Analyse durchzuf�hren. \\
Bei der Studie handelt es sich um sieben Projekte, aus verschiedenen Dom�nen, welche in der folgenden Tabelle aufgef�hrt sind:

 \begin{figure}[htbp] 
	\centering
	\includegraphics[width=1.2\textwidth]{images/table.png}
	\caption{Untersuchte Projekte (Abb. aus \cite{kour}, S. 251, Table 1)}
	\label{fig:table}
\end{figure}

Eine Analyse ist nur m�glich, wenn Daten zur Verf�gung stehen. Daf�r wurden gleiche Zeitr�ume betrachtet und �nderungen und bug fixes analysiert, um dann die Ergebnisse miteinander zu vergleichen (\cite{kour}, S. 251). Bei den Ergebnissen des hier zitierten Papers handelt es sich um Daten aus dreimonatigen Intervallen.

\section{Zusammenfassung der Ergebnisse der Studie}
Wie bereits dargestellt, handelte es sich bei der aufgef�hrten Studie um eine empirische Analyse von Daten. \\
Bei der Zusammenfassung der Ergebnisse werden die drei zuvor aufgestellten Forschungsfragen aufgegriffen und mithilfe der Ergebnisse beantwortet. 

\subsection{Forschungsfrage 1}
Folgende Tabelle fasst die f�nf Modelle \textit{Bunch, ArchDRH, ACDC, Package }und \textit{LDA view} zusammen, die sich auf die erste Forschungsfrage beziehen (\cite{kour}, S. 252).

 \begin{figure}[htbp] 
	\centering
	\includegraphics[width=1.2\textwidth]{images/table2.png}
	\caption{Ergebnisse der verschiedenen Views (Abb. aus \cite{kour}, S. 252, Table 2)}
	\label{fig:table2}
\end{figure}

Da es keinen Zugang zu dem Quellcode des kommerziellen Projektes gibt, k�nnen die Daten des \glqq \textit{LDA view}\grqq nicht aufgef�hrt werden (\cite{kour}, S. 252).
Die Daten zeigen, dass die aufgestellte Proposition - dass modul�bergreifende �nderungen fehleranf�lliger sind, als modulinterne �nderungen - best�tigt werden kann (\cite{kour}, S. 252). Somit l�sst sich Frage 1 mit zutreffend beantworten. Zus�tzlich konnten keine Unterschiede zwischen Open Source Projekten und kommerziellen Projekten festgestellt werden (\cite{kour}, S. 252).

\subsection{Forschungsfrage 2}
Alle Projekte, die in der Studie untersucht wurden, zeigen, dass modul�bergreifende �nderungen einen st�rkeren Einfluss auf \textit{bugs} haben, als �nderungen, die innerhalb eines Moduls stattfinden (\cite{kour}, S. 252). Es l�sst sich anhand der Daten ebenfalls feststellen, dass Entwickler alle verf�gbaren \glqq \textit{surrogate views}\grqq, au�er die \textit{high-level packages}, nutzen k�nnen, um die �nderungen innerhalb des Systems zu �berwachsen (\cite{kour}, S. 253). Im Endeffekt bedeutet dies, dass man sogar \textit{low-level packages }nutzen kann, um die Gesundheit des Systems und deren �nderungshistorie zu �berwachen, anstatt komplexe Methoden anzuwenden. Zusammenfassend l�sst sich deshalb die zweite Forschungsfrage damit beantworten, dass keine \textit{surrogate view} besser als eine andere ist, ausgenommen ist die\textit{ high-level package view}, da sie gleiche Resultate bez�glich des Verh�ltnisses zwischen �nderungsverteilung und Fehlern aufzeigen (\cite{kour}, S. 253).

\subsection{Forschungsfrage 3}
Die Studie hat herausgefunden, dass die Dimension der �nderung (zum Beispiel modul�bergreifende versus modulinterne �nderungen) wichtiger ist als der Aufwand der �nderung (\cite{kour}, S. 253). Weiterhin zeigt die Studie, dass wenn man eine Metrik benutzt, die modul�bergreifende �nderungen differenziert, es zu Verbesserungen f�hrt\textit{ bugs} fehlerfrei vorherzusagen (\cite{kour}, S. 253). Abschlie�end l�sst sich Forschungsfrage drei dahingehend beantworten, dass eine �nderungsmetrik, die Architekturmodule ber�cksichtigt, h�here Korrelationen mit Fehlern aufweist als eine, die keine modul�bergreifenden �nderungen differenziert (\cite{kour}, S. 253).
\section{Diskussion}
Wie eingangs erw�hnt, ist es bereits bekannt, dass die Softwarearchitektur eine wichtige Rolle in Bezug auf die Evolution spielt, besonders in Bezug auf die �nderungen, die innerhalb des Systems vollzogen werden k�nnen (\cite{kour}, S. 253). Die Studie hat empirische Belege herausgefunden, die die Wichtigkeit der Softwarearchitektur in der Evolution eines Softwaresystems beweisen. Wie bereits erw�hnt, kennen sich die Entwickler meist nur mit einem kleinen Teil der Module aus, gerade wenn es sich um ein gro�es Softwaresystem handelt. So ist es f�r den Entwickler noch schwieriger, die Konsequenzen der �nderungen und das Verhalten des Systems zu verstehen, weshalb es wahrscheinlicher ist, bei modul�bergreifenden �nderungen Fehler zu machen (\cite{kour}, S. 254).
Die Studie bekr�ftigt zus�tzlich die g�ngige Meinung, dass die Entscheidungen der Softwarearchitektur (zum Beispiel wie sich ein Softwaresystem in seine Module zerlegen l�sst) einen signifikanten Einfluss auf die Evolution des Systems hat (\cite{kour}, S. 254). So wird auch der Einfluss von Softwarearchitektur auf Open Source Projekte deutlich. Die Studie zeigt demnach auf, wie wichtig es w�re, dass auch die Open Source Entwickler die Softwarearchitektur dokumentieren (\cite{kour}, S. 254).
Interessant f�r zuk�nftige Studien w�re es, die Programmiersprache von Java in eine andere zu wechseln, wie zum Beispiel C++, allerdings m�sste dann der Begriff \textit{package view} umdefiniert werden, der ausschlie�lich in dieser Studie f�r Java definiert worden ist, allerdings ist es m�glich �hnliche Konzepte f�r diesen Begriff in anderen Programmiersprachen zu definieren  (\cite{kour}, S. 255).

\section{Zusammenfassung der Studie}
Abschlie�end l�sst sich zu dieser Studie sagen, dass sie die Wichtigkeit der Softwarearchitektur, in Bezug auf Design und Wartung von Software herausgearbeitet hat (\cite{kour}, S. 255). Jedoch ger�t dies in der Praxis, vor allem durch die\textit{ Open Source community}, oftmals in Vergessenheit. Diese nutzen nicht die Vorteile von Softwarearchitekturprinzipien, die das System zusammenhalten. Auch wenn sie nicht explizit diese Prinzipien vermeiden, stehen diese nicht im Vordergrund und werden w�hrend der Systemevolution weder repr�sentiert, noch aufrechterhalten (\cite{kour}, S. 255). Die Studie zeigt empirisch, wie sich das Benutzen einer Softwarearchitektur w�hrend der Evolution auf die Qualit�t der Software im Ganzen auswirkt. Herausgefunden wurde, dass �nderungen, die modul�bergreifend sind, fehleranf�lliger sind als jene, die lediglich modulintern vollzogen werden (\cite{kour}, S. 255). Dies gilt sowohl f�r kommerzielle Projekte als auch f�r Open Source Projekte mit dokumentierten Softwarearchitekturen. Die Studie bekr�ftigt, die Wichtigkeit der Ber�cksichtigung einer Softwarearchitektur, da sie als eine der Schl�sselfaktoren, bezogen auf die Qualit�t f�r das Ver�ndern von Softwaresystemen, gilt (\cite{kour}, S. 255). 





\chapter{Architekturtaktiken}
In diesem Kapitel werden Architekturtaktiken vorgestellt, die bei Anwendung der Evolutionsperspektive im vierten Schritt eingesetzt werden k�nnen, um die bestehende Architektur zu �berarbeiten. Nach \cite{SSA:12} ist eine Architekturtaktik definiert als \glqq eine etablierte und bew�hrte Vorgehensweise, die verwendet werden kann, um eine bestimmte Qualit�tseigenschaft zu erreichen.\grqq (\cite{SSA:12}, S.48, Z.11-12, eigene �bersetzung). Die Qualit�tseigenschaften sind in diesem Fall alle f�r die Evolutionsperspektive wichtigen Concerns.


\section{Begriffe}
In diesem Abschnitt werden zun�chst die zum Verst�ndnis dieses Kapitel notwendigen Begriffe erl�utert.

\paragraph{Modul}
Ein Modul ist ein zusammenh�ngender Satz von Funktionen und Datenstrukturen (\cite{CA:18}, S.86). 
Im Falle der objektorientierten Programmierung ist ein Modul beispielsweise eine Klasse oder ein Interface.
\paragraph{Komponente}
Komponenten sind als Deployment-Einheiten zu verstehen, d.h. \glqq sie repr�sentieren die kleinsten Entit�ten, die als Teil eines Systems deployt werden k�nnen\grqq (\cite{CA:18}, S.113, Z.2-3). Es ist bei gutem Komponentendesign m�glich, Komponenten unabh�ngig voneinander zu entwickeln und zu deployen, beispielsweise als Plug-ins im .jar oder .exe Format (\cite{CA:18}, S.113). Diese Unabh�ngigkeit erleichtert den Umgang mit �nderungen enorm, da Entwickler entscheiden k�nnen, wann ge�nderte Komponenten integriert werden (\cite{CA:18}, S.131).

\paragraph{Abh�ngigkeit}
Wenn ein Modul ein anderes Modul verwendet, ist dies eine Quellcode-Abh�ngigkeit, kurz Abh�ngigkeit, vom verwendeten Modul. 
Jede Objekterzeugung stellt eine Abh�ngigkeit von der konkreten Definition des Objektes dar (\cite{CA:18}, S.109). Die st�rkste und strengste Abh�ngigkeit stellt Vererbung dar (\cite{CA:18}, S.108).

\paragraph{Stabilit�t}
Stabilit�t im Zusammenhang mit Software gibt an, wie viel Aufwand erforderlich ist, ein Modul bzw. eine Komponente zu �ndern. Je gr��er der Aufwand, desto stabiler ist sie. Eine Komponente ist sehr stabil, wenn sie viele eingehende , aber kaum ausgehende Abh�ngigkeiten aufweist (\cite{CA:18}, S. 138).

\paragraph{Design-Pattern}
\glqq Ein Design-Pattern dokumentiert eine oft wiederkehrende und etablierte Struktur von miteinander verbundenen Design-Elementen, die ein generelles Design Problem in einem bestimmten Kontext l�st\grqq (\cite{SSA:12}, S.165, Z.4-6, eigene �bersetzung). Design-Patterns beziehen sich demnach immer auf ein Problem in einem Kontext.
\paragraph{Designprinzip}
\glqq Ein Software-Designprinzip ist eine umfassende und fundamentale Doktrin oder Regel, welche die Entwicklung von qualitativen Software Designs leitet\grqq (\cite{princ}, S.1, Z.64-66, eigene �bersetzung). 
Designprinzipien sind allgemeiner als Patterns, d.h. sie sind nicht an ein Problem oder einen Kontext gebunden und unabh�ngig von Implementierungsdetails.



\section{Designprinzipien}
Die erste hier vorgestellte Architekturtaktik besteht darin, etablierte Designprinzipien zu verwenden. Sie helfen, Auswirkungen von �nderungen einzuschr�nken (\cite{SSA:12}, S.553) sowie die Komplexit�t des Systems zu verringern. Eine verst�ndliche und gut strukturierte Architektur l�sst sich einfacher erweitern und verbessern. \\

 
\subsection{Die SOLID-Prinzipien}
Bei den SOLID-Prinzipien handelt es sich um Designprinzipien, welche sich auf die Modulebene beziehen. Der Begriff SOLID ist ein Akronym f�r die einzelnen Prinzipien, wobei der Begriff um 2004 von Robert C. Martin gepr�gt wurde. Ihm ist die Zusammenstellung der Prinzipien in ihrer heutigen Form zu verdanken. Die Prinzipien selbst reichen jedoch weitaus l�nger zur�ck, wennngleich sie sich im Laufe der Jahre immer wieder ver�ndert haben. Die SOLID-Prinzipien gelten nicht nur f�r objektorientierte Programmierung (\cite{CA:18}, S.82-83). \\
Ziel der SOLID-Prinzipien ist es, Module so zu gestalten und organisieren, dass diese �nderungen tolerieren und leicht nachvollziehbar sind. Damit wird �nderbarkeit bereits auf Modulebene unterst�tzt und es wird das Fundament f�r gut struktuierte Komponenten gelegt. 



\subsubsection{Das Single-Responsibility-Princip (SRP)}
Das Single-Responsibility-Prinzip wird aufgrund des Namens oft missverstanden, mit der Annahme, dass jedes Modul nur eine Aufgabe haben soll. Dies beschreibt jedoch das Separation of Concerns Prinzip. 
Das Single-Responsibility-Prinzip lautet hingegen wie folgt:
\glqq Ein Modul sollte f�r einen, und nur einen, Akteur verantwortlich sein\grqq (\cite{CA:18}, S.86, Z.18). Daraus folgt, dass \glqq Code, von dem verschiedene Akteure abh�ngig sind, separiert werden muss\grqq (\cite{CA:18}, S.88, Z.19-20). \\
Akteur ist hierbei ein Sammelbegriff f�r Gruppen, die gemeinsame �nderungsinteressen haben. Das kann ein User oder Stakeholder sein, aber auch mehrere. Das SRP soll verhindern, dass �nderungen f�r einen Akteur sich unbeabsichtigt und eventuell sogar unbemerkt auf andere Akteure auswirken (\cite{CA:18}, S.85-88). 

\subsubsection{Das Open-Closed-Prinzip (OCP)}
\label{ocp}
Das Open-Closed-Prinzip wurde 1988 von Bertrand Meyer formuliert (\cite{meyer}, S.23) und besagt: \glqq Eine Softwareentit�t sollte offen f�r Erweiterungen, aber zugleich auch geschlossen gegen�ber Modifikationen sein.\grqq (\cite{CA:18}, S.91, Z.4-5). \\
Die M�glichkeit, Module erweitern zu k�nnen, ohne bestehenden Code ver�ndern zu m�ssen, stellt den Idealfall dar. Das OCP sollte darum stets ein leitendes Motiv beim Entwurf von Modulen sein.
\subsubsection{Das Liskov'sche Substitutionsprinzip (LSP)}
\label{lsp}
Das Liskov'sche Substitutionsprinzip wurde 1987 von Barbara Liskov formuliert \cite{barbara}.
Es besagt �bersetzt: \glqq Was hier erreicht werden sollte, ist etwas wie die folgende Substitutionseigenschaft: Wenn f�r jedes Objekt o1 vom Typ S ein Objekt o2 vom Typ T existiert, sodass f�r alle Programme P, die in T definiert sind, das Verhalten von P unver�ndert bleibt, wenn o1 f�r o2 substituiert wird, dann ist S ein Subtyp von T\grqq (\cite{CA:18}, S.97, Z.5-8).\\
Einfacher ausgedr�ckt lautet die Aussage: S ist ein Subtyp von T, wenn T durch S ersetzt werden kann und das Programmverhalten weiterhin gleich bleibt.
Es ist also gefordert, dass Module durch ihre Subtypen wechselseitig ersetzbar sind. Das LSP spielt vor allem bei Vererbung eine wesentliche Rolle. Wenn das LSP eingehalten wird, kann ein Modul der Klasse T von allen Klassen, die von T erben, substituiert werden, was ein hohes Ma� an Flexibilit�t erlaubt. Wenn T ein Interface ist, dann kann T von allen Klassen substituiert werden, welche das Interface implementierten. Umgekehrt wird das LSP verletzt, wenn die Unterklassen keine echten Unterklassen sind und sich das Systemverhalten bei einem Austausch �ndert(\cite{CA:18}, S.98-99).


\subsubsection{Das Interface-Segregation-Prinzip (ISP)}
Das Interface-Segregation-Prinzip besagt, dass Abh�ngigkeiten von nicht genutzten Modulen vermieden werden sollten (\cite{CA:18}, S.84, Z.15-16). \\
Beispielsweise stellen transitive Abh�ngigkeiten eine solche Abh�ngigkeit dar, die es gilt aufzul�sen (\cite{CA:18}, S.105). 
Es kann sie auch auf kleinerer Ebene in Form von ungenutzten Funktionen geben. Wenn eine Klasse viele Funktionen enth�lt, aber nur wenige davon tats�chlich ben�tigt werden, macht es Sinn, ein Interface dazwischenzuschalten, welches nur die ben�tigten Funktionen enth�lt. Auf diese Weise haben �nderungen an ungenutzten Funktionen keine Auswirkungen mehr. Werden f�r verschiedene Klassen, die unterschiedliche Funktionen nutzen, jeweils ein solches Interface dazwischengeschaltet, erh�lt man eine Auftrennung der Funktionen durch Interfaces, was den Namen des Prinzips begr�ndet (\cite{CA:18}, S.103-104).
%TODO
%Abbildung?


\subsubsection{Das Dependency-Inversion-Prinzip (DIP)}
\label{dip}
Das f�nfte der SOLID-Prinzipen besagt: \glqq Der Code, der die �bergeordneten Richtlinien (engl. Policy) implementiert, sollte nicht von dem Code abh�ngig sein, der untergeordnete Details implementiert. Vielmehr sollten Details von den Richtlinien abh�ngig sein.\grqq (\cite{CA:18}, S.84, Z.17-21).  \\
Abh�ngigkeiten von Modulen, die sich oft �ndern, sind zu vermeiden. In diesem Fall ist es vorzuziehen, eine Abstraktion zwischen den Modulen einzubauen, sodass sowohl die stabilen, �bergeordneten als auch die untergeordneten Module beide von der Abstraktion abh�ngig sind (\cite{CA:18}, S.107-108). Bei der Abstraktion kann es sich beispielsweise um ein Interface oder eine abstrakte Klasse handeln.\\
Eine alternative Defintion des DIP lautet deshalb: \glqq Unser Entwurf soll sich auf Abstraktionen st�tzen. Er soll sich nicht auf Spezialisierungen st�tzen.\grqq (\cite{OOP:09}, Z.12-13). \\
Durch Quellcode-Abh�ngigkeiten, die sich aussschlie�lich auf Abstraktionen beziehen, ist das System sehr flexibel. Die fl�chtigen Module, die untergeordnete Details enthalten, k�nnen sich beliebig oft �ndern, ohne dass die �bergeordneten Module davon beeinflusst werden, da die Abstraktion in den meisten F�llen unver�ndert bleibt.
Beispielsweise bleibt ein Interface meist unbeeinflusst, wenn sich eine Klasse �ndert, die das Interface implementiert.  \\
Damit das DIP funktioniert, muss allerdings bewusst darauf geachtet werden, dass die abstrakten Schnittstellen so stabil wie m�glich gestaltet werden, denn mit fl�chtigen Schnittstellen ist nichts gewonnen (\cite{CA:18}, S.108). \\
�berall Abstraktionen einzubauen ist in der Praxis jedoch nicht realistisch. Nicht alle Abh�ngigkeiten zu konkreten Modulen lassen sich verhindern. Es ist hier wichtig, abzuw�gen, wann Abh�ngigkeiten akzeptabel sind. Module, die stabil sind, werden eher unwahrscheinlich ge�ndert, hier sind Abh�ngigkeiten tolerierbar (\cite{CA:18}, S.107-108). Zudem k�nnen die Auswirkungen von Abh�ngigkeiten abgeschw�cht werden, indem die konrekten Module gemeinsam in konkreten Komponenten gruppiert werden, sodass die Abh�ngigkeiten lokal begrenzt und vom restlichen System getrennt sind (\cite{CA:18}, S.110).\\
Ein Beispiel f�r das DIP erfolgt an sp�terer Stelle in diesem Kapitel (siehe Abschnitt \ref{example}).

\subsection{Weitere Designprinzipien}
Im Folgendenn werden einige weitere Designprinzipien erl�utert, die nicht zu den SOLID-Prinzipien geh�ren, aber ebenso dazu beitragen, die Auswirkungen von �nderungen lokal einzuschr�nken (\cite{SSA:12}, S.553). 
\subsubsection{Encapsulation}
Encapsulation oder Kapselung bechreibt in der objektorientierten Programmierung die Vorgehensweise, klasseninterne Daten und Funktionen mit dem kleinstm�glichen Zugriffsrecht zu versehen und lediglich �ber eine �ffentliche Schnittstelle zur Verf�gung zu stellen, damit diese vor Zugriffen von au�en gesch�tzt sind. Beispielsweise k�nnen Variablen auf \textit{private} gesetzt und lediglich �ber Getter- und Settermethoden zug�nglich gemacht werden (\cite{CC:09}, S.136).\\
Dies sorgt f�r eine geringe Kopplung zwischen den Klassen, da auf die Daten nicht uneingeschr�nkt zugegriffen werden kann. Je weniger andere Klassen mit den Daten oder Funktionen in Ber�hrung kommen, desto weniger wirken sich �nderungen daran auf die anderen Klassen aus \cite{kaps}.

\subsubsection{Separation of Concerns}
Das Separation of Concerns Prinzip besagt, dass jedes Systemelement eine klare Verantwortlichkeit haben sollte. Ein Element kann beispielsweise eine Funktion, ein Modul oder eine ganze Komponente sein. Separation of Concerns sollte auf allen Ebenen eingehalten werden. Wird das Prinzip befolgt, wirken sich �nderungen an einem Element nur lokal eingeschr�nkt aus, wird es hingegen missachtet, k�nnen bereits bei kleinsten �nderungen weite Teile des Systems mitbetroffen sein (\cite{SSA:12}, S.553).


\subsubsection{Funktionale Koh�sion}
Funktionale Koh�sion gibt an, wie stark die Funktionen eines Moduls oder einer Komponente miteinander in Beziehung stehen und damit logisch zusammengeh�ren. Demnach spricht eine hohe Koh�sion daf�r, dass das System sinnvoll in Module bzw. Komponenten unterteilt ist. Bei einer hohen funktionalen Koh�sion wirken �nderungen sich meist nur lokal auf jenen zusammenh�ngenden Bereich aus (\cite{SSA:12}, S.553).
Die konsequente Befolgung dieses Prinzips f�hrt zu vielen kleinen Modulen(\cite{CC:09}, S.140). 

\subsubsection{Single Point of Definition}
Das Single Point of Definition Prinzip besagt, dass Datentypen, Werte, Algorithmen, Konfigurationen Schemata etc. nur einmal definiert und implementiert werden sollen (\cite{SSA:12}, S.553). Es gibt demnach stets genau einen Definitionspunkt. Dies bietet den Vorteil, dass jene Elemente bei einer Anpassung nur einmal ge�ndert werden m�ssen, n�mlich an der Stelle, an der sie definiert sind.



\subsection{Komponentenprinzipien}
Im Folgenden werden weitere Designprinzipien vorgestellt, die sich jedoch nicht mehr auf die Modul-, sondern ausschlie�lich auf die Komponentenebene beziehen.

 
 \subsubsection{Das Acyclic-Dependencies-Prinzip (ADP)}
 Das Acyclic-Dependencies-Prinzip bezieht sich auf die Komponentenkopplung, d.h. die Beziehungen zwischen Komponenten. Das Prinzip besagt, dass im Schema der Komponentenabh�ngigkeiten keine Zyklen auftreten d�rfen (\cite{CA:18}, S.129, Z.8). \\
 %Wird eine  Komponente A ge�ndert, m�ssen alle Komponenten, die von A abh�ngen, an die neue Version von A angepasst werden. Es ist auch klar, zu welchen Komponenten A kompatibel sein muss, n�mlich zu allen, von denen A selbst abh�ngt. \\
 Die Auswirkungen von �nderungen lassen sich nicht mehr klar absch�tzen, sobald ein Abh�ngigkeitszyklus vorliegt. 
  Abbildung \ref{fig:adp} zeigt einen solchen Abh�ngigkeitszyklus in einem f�r Anwendungen typischen Komponentendiagramm (\cite{CA:18}, S.131). \\
  Durch die zyklische Abh�ngigkeit verschmelzen die Komponenten \textit{Entities}, \textit{Authorizer} und \textit{Interactors} praktisch zu einer einzigen Komponente, obwohl sie unabh�ngig voneinander sein sollen. Wenn \textit{Authorizer} sich �ndert, muss nicht nur \textit{Entities} angepasst werden, um kompatibel zu sein, sondern auch \textit{Interactors}, da die Komponente von \textit{Entities} und damit transitiv von \textit{Authorizer} abh�ngt.
  Dies gilt analog f�r \textit{Interactors} und \textit{Entities}. Die durch den Zyklus entstandenen transitiven Abh�ngikeiten erschweren sowohl das Entwickeln als auch das Testen. Es gibt keine richtige Reihenfolge mehr, in der die Komponenten erstellt oder ge�ndert werden sollten (\cite{CA:18}, S.133-134). Da ein Zyklus keinen Endpunkt hat, kann es in der Theorie eine endlose Folge von notwendigen Anpassungen geben. In der Praxis werden die Anpassungen zwar nicht endlos sein, aber unangenehm und vor allem vermeidbar aufwendig.
 
 \paragraph{Aufl�sung mittels DIP}
 \label{example}
 Der Zyklus kann mittels Anwendung des Dependency-Inversion-Prinzips unterbrochen werden. Es wird ein Interface erzeugt, in dem alle Methoden enthalten sind, die \textit{User} ben�tigt. Dieses Interface geh�rt zur Komponente \textit{Entities}. User h�ngt von dem Interface ab, aber die Abh�ngigkeit ist innerhalb derselben Komponente und geht nun nicht mehr zu \textit{Authorizer}. \textit{Authorizer} implementiert das Interface und ist nun vom Interface und damit von Entities abh�ngig. Damit zeigt der Abh�ngigkeitspfeil in die umgekehrte Richtung und der Zyklus wurde unterbrochen. Abbildung \ref{fig:loes1_adp} verdeutlicht dieses Vorgehen.
 
 
 \begin{figure}[htbp] 
 	\centering
 	\includegraphics[width=1\textwidth]{images/adp.png}
 	\caption{Ein Abh�ngigkeitszyklus(basierend auf \cite{CA:18}, S.134 Abb. 14.2. sowie S.135, Abb.14.3)}
 	\label{fig:adp}
 \end{figure}
 
 
 \begin{figure}[htbp] 
 	\centering
 	\includegraphics[width=1\textwidth]{images/adp_2.png}
 	\caption{Aufl�sung des Abh�ngigkeitszyklus mittels DIP (basierend auf \cite{CA:18}, S.134 Abb. 14.2. sowie S.135, Abb.14.3)}
 	\label{fig:loes1_adp}
 \end{figure}
 
 
 

%\subsubsection{Das Reuse-Release-Equivalence-Prinzip (REP)}
%\subsubsection{Das Common-Closure-Prinzip (CCP)}
%\subsubsection{Das Common-Reuse Prinzip (CRP)}




\subsubsection{Das Stable-Dependencies-Prinzip (SDP)}
\label{sdp}
Das Stable-Dependencies-Prinzip bezieht sich auf die Komponentenkopplung  und lautet: \glqq Abh�ngigkeiten sollten in dieselbe Richtung verlaufen wie sie Stabilit�t.\grqq (\cite{CA:18}, S.137, Z.16). \\
Demnach sollen stabile Komponenten nicht von instabilen Komponenten abh�ngen, die sich mit hoher Wahrscheinlichkeit �ndern werden.
Stabile Komponenten lassen sich nur schwer modifizieren und behindern damit die �nderungen an der instabilen Komponente, da sie nur schwer daran angepasst werden k�nnen. Stattdessen sollen die instabilen von den stabilen Komponenten abh�ngen\cite{CA:18}, S.137).\\
Die Instabilit�t einer Komponente $I \in [0,1]$ l�sst sich wie in Formel \ref{eq} angegeben bestimmen (\cite{CA:18}, S.139), wobei 0 maximal stabil entspricht und $\#$ ist die Anzahl bezeichnet. Verst��e gegen das SDP lassen sich mit dem DIP (siehe \ref{dip}) l�sen (\cite{CA:18}, S.139-142).


\begin{equation}
\label{eq}
	I=\frac{\#\, ausgehende\, Abh\ddot{a}ngigkeiten}{\#\, eingehende\, Abh\ddot{a}ngigkeiten +  \#\, ausgehende\, Abh\ddot{a}ngigkeiten}
\end{equation}



\subsubsection{Das Stable-Abstractions-Prinzip (SAP)}
Das Stable-Abstractions-Prinzip bezieht sich ebenfalls auf die Komponentenkopplung und lautet:
\glqq Eine Komponente sollte ebenso abstrakt sein, wie sie stabil ist.\grqq (\cite{CA:18}, S.143, Z.14). \\
Damit sollen stabile Komponenten aus Schnittstellen sowie abstrakten Klassen und instabile Komponenten aus konkreten Klassen bestehen. Diese Abstraktion erlaubt es, stabile Komponenten trotz ihrer Stabilit�t erweitern zu k�nnen.
SAP und SDP zusammengenommen ergeben das DIP auf Komponentenebene. Der wesentliche Unterschied ist, dass Komponenten im Gegensatz zu Modulen nicht entweder abstrakt oder konkret sind, sondern sich auch dazwischen befinden k�nnen.
Der Grad der Abstraktion einer Komponente $A \in [0.1]$ l�sst sich genau wie Instabilit�t als Softwaremetrik messen mittels der in \ref{eq2} angegebenen Formel (\cite{CA:18}, S.144), wobei 1 f�r maximal abstrakt steht.
\begin{equation}
\label{eq2}
A=\frac{\#\, abstrakte\,Klassen\,oder\,Schnittstellen}{\#\, Gesamtzahl\,Klassen}
\end{equation}



\section{Design-Patterns}
Diese Architekturtaktik besteht darin, etablierte Design-Patterns zu verwenden, die �nderungen vereinfachen.
Es exisitiert eine breite Auswahl solcher Patterns. Im Folgenden werden Beispiele aufgef�hrt.


\subsection{Abstract Factory}
Das DIP (siehe Abschnitt \ref{dip} bedingt, dass konkrete, fl�chtige Objekte nicht ohne Weiteres erzeugt werden k�nnen, denn jede Objekterzeugung stellt eine Abh�ngigkeit zu einer konkreten Klasse des Objektes dar (\cite{CA:18}, S.109). 
Abhilfe schafft das Design-Pattern Abstract Factory. Dieses Pattern erlaubt es, Familien verwandter Objekte zu erzeugen, ohne deren konkrete Klassen zu spezifizieren (\cite{pat}, S.90). \\
In Abbildung \ref{fig:fact} ist das Pattern am Beispiel einer plattform�bergreifenden grafischen Anwendung gezeigt. In der Mitte befindet sich das Interface \textit{GUIFactory}, dies ist die Abstract Factory. Hier werden Methoden f�r die konkreten Factories vorgegeben. \textit{Windows Factory} und \textit{MacFactory} sind die konkreten Factories, sie implementieren die Methoden der Abstract Factory. Das Interface \textit{Checkbox }stellt ein abstraktes Produkt dar, das von den konkreten Produkten \textit{WindowsCheckbox} und \textit{MacCheckbox} implementiert wird, analog verh�lt es sich mit dem Interface \textit{Button} (\cite{pat}, S.96). \\
Der erste Vorteil besteht in der Konsistenz der Produkte: Die \textit{WindowsFactory} stellt nur zueinander passende Windows UI-Elemente her, analog bei der \textit{MacFactory}. Das Pattern l�sst sich beispielsweise um eine Linux-Factory erweitern, die konsistente Linux UI-Elemente enth�lt. \\
Der zweite Vorteil besteht darin, dass die Anwendung lediglich von \textit{GUIFactory} abh�ngig ist, d.h. von einer Abstraktion, was dem DIP entspricht. \\
Es stellt sich die Frage, an welcher Stelle eine konkrete Factory-Instanz erzeugt wird, z.B. eine \textit{WindowsFactory}. Dies geschieht  in einer konkreten Klasse innerhalb einer konkreten Komponente, wie z.B. in der \textit{main}-Methode, wobei die erzeugte Instantz in einer globalen Variablen gespeichert wird, auf welche die Anwendung dann global zugreifen kann (\cite{CA:18}, S.110).


 \begin{figure}[htbp] 
	\centering
	\includegraphics[width=1\textwidth]{images/abstract_factory.png}
	\caption{Design Pattern Abstract Factory (Abbildung basierend auf \cite{pat}, S.97)}
	\label{fig:fact}
\end{figure}


\subsection{Dependency Injection}
Dieses Design-Pattern hilft, �bergeordnete Module vor den Implementierungsdetails von untergeordneten Modulen zu sch�tzen (\cite{SSA:12}, S.555). \\
Abbildung \ref{fig:dep_inj} zeigt Constructor Dependency Injection an einem konkreten Beispiel. Die Klasse \textit{PaymentTerms} beinhaltet alle Informationen, die zur Berechnung der monatlichen Kosten eines Kredits notwendig sind. Die Klasse \textit{PaymentCalculator} enth�lt die Methode \textit{getMonthlyPayment()}, d.h. hier wird die konkrete Berechnung durchgef�hrt. Das Problem besteht darin, dass \textit{PaymentTerms} diese Methode aufrufen muss und ohne Dependency Injection direkt von \textit{PaymentCalculator} abh�ngig w�re. Dies w�re bedenklich: Es k�nnnten neue Arten von \textit{Calculator}-Klassen hinzukommen, welche die Berechnung anders durchf�hren, oder vielleicht gibt es auch bereits mehrere solcher \textit{Calculator}-Klassen. Dann sollte es nicht in der Klasse\textit{ PaymentTerms} entschieden werden, welche Berechnung verwendet wird, sondern in einem �bergeordneten Modul. \\
Constructor Dependency Injection l�st dies folgenderma�en: Es wird ein Interface erzeugt, hier \textit{IPaymentCalculator} genannt, das die ben�tigte Methode enth�lt und das von \textit{PaymentCalculator }und allen m�glichen weiteren \textit{Calculator}-Klassen implementiert wird. Der Klasse \textit{PaymentTerms} wird im Konstruktor ein\textit{ IPaymentCalculator}-Objekt �bergeben. Damit besteht nur noch eine Abh�ngigkeit zu einer Abstraktion, was dem DIP entspricht (siehe \ref{dip}). Der konrkete Calculator-Typ wird erst zur Laufzeit entschieden, da die Abh�ngigkeit �ber den Konstruktor injected wird, was den Namen des Patterns erkl�rt. Das bietet eine hohe Flexibilit�t, denn die Implementierungsdetails in \textit{PaymentTerms} stellen nun kein Problem mehr da. Wird ein anderer Berechnungstyp ben�tigt, kann von einem �bergeordneten Modul aus einfach ein anderes \textit{IPaymentCalculator}-Objekt �bergeben werden. Es ist anzumerken, dass Constructor Injection nur ein m�glicher Typ von Dependency Injection ist. Es existieren zudem noch Setter Injection und Interface Injection \cite{dep}. Dependency Injection ist auch unter dem Namen Inversion of Control bekannt (\cite{SSA:12}, S.555).


 \begin{figure}[htbp] 
	\centering
	\includegraphics[width=1\textwidth]{images/dep_inj.png}
	\caption{Constructor Dependency Injection (Abbildung basierend auf \cite{dep})}
	\label{fig:dep_inj}
\end{figure}


\section{Erweiterbare Interfaces}
Diese Architekturtaktik bezieht sich auf die Gestaltung von Interfaces. Ihnen sollte besondere Aufmerksamkeit zukommen, denn �nderungen an Interfaces haben den gr��ten Einfluss auf das Systen und verursachen damit die gr��ten Kosten. Wird beispielsweise ein Parameter einer Funktion ge�ndert, m�ssen alle Klassen angepasst werden, welche das Interface implementieren und damit diese Funktion verwenden. \\
Eine m�gliche L�sung besteht darin, anstatt vieler Paramter Objekte oder andere strukturierte Datentypen in der Funktion zu �bergeben. Alle Parameter sind dann als Attribute in der Klasse des Objekts vorhanden. Der Vorteil ist, dass Attribute sich optional gestalten lassen, indem sie mit Default-Werten initialsiert werden.\\
Analog verh�lt es sich mit Information Interfaces. Anstelle von Objekten k�nnen selbstbeschreibende Nachrichtenformate �bergeben werden, wie z.B. XML. Wenn das Nachrichtenformat es zul�sst, wie XML, k�nnen Elemente optional gesetzt werden. \\
Eine solch flexible Gestaltung von Interfaces geht zulasten von Verst�ndlichkeit und Testbarkeit, da es beispielsweise nur schwer auff�llt, wenn Elemente fehlen. Es kann zudem auch zu Performanzeinbu�en kommen. Darum gilt es abzuw�gen, welche Interfaces in Zukunft wahrscheinlich erweitert werden und eine solche Flexibilit�t ben�tigen (\cite{SSA:12}, S.553-554). 


\section{Metamodell-basierte Architekturstile}
Diese Architekturtaktik besteht darin, metamodell-basierte Architekturstile zu verwenden. Hierbei handelt es sich um eine sehr tiefgreifende Taktik, da der gesamte Architekturstil auf Evolution ausgerichtet wird (\cite{SSA:12}, S.555). Man bezeichnet sie auch als Metasysteme, da es sich um Systeme von Systemen handelt (\cite{meta}, S.1). \\
Metamodell-basierten Architekturstile ist ein �bergeordnetes Metamodell gemeinsam, welches das Zusammenspiel der Komponenten koordiniert. Die Konfigurationen des Metamodells entscheiden, wie die Komponenten zur Laufzeit zusammengesetzt werden. Oft reicht es, bei �nderungen die Konfigurationen des Metamodells anzupassen. Da die Konfigurationen jedoch zur Laufzeit eingelesen und umgesetzt werden, sind Metasysteme im Hinblick auf Performanz eingeschr�nkt. Der hohe Grad an Flexibilit�t schl�gt sich zudem auch in einer erh�hten Komplexit�t nieder, die sowohl das Entwickeln als auch das Testen erschwert (\cite{SSA:12}, S.556).\\
Es gibt verschiedene Arten von Metasystemen. Es sei an dieser Stelle ein Beispiel genannt. \\
Nach \cite{meta} ist ein Metasystem ein \glqq gro� angelegtes, verteiltes System, dessen Komponenten Enterprise-Systeme sind, die durch den Governance Mechanismus des Metamodells miteinander verkn�pft sind, um ein gemeinsames strategisches Ziel zu erreichen\grqq (\cite{meta}, S.1, Z.66-73, eigene �bersetzung). Das �bergeordnete Governance System leitet, �berwacht und koordiniert alle Operationen des Metasystems. Die untergeordneten Enterprise-Systeme, d.h. die Komponenten des Systems, k�nnen bei Bedarf rekonfiguriert werden, ohne dass die Implementierung des Systems angepasst werden muss. Zudem lassen sich die Enterprise-Systeme austauschen, es lassen sich zudem Systeme entfernen und neue hinzuf�gen. Damit ist diese Architektur hochflexibel und passt sich einer sich �ndernden Umgebung in co-evolution�rer Weise an (\cite{meta}, S.2). Anwendung findet ein solcher Architekturstil beispielsweise bei gro�en Energieverteilungs-, Informations- oder Kommunikationsnetzwerken oder auch bei Netzwerken, welche die Luftfahrt kontrollieren. Dort m�ssen die Systeme sich an st�ndig �ndernde Informationen und Bedingungen anpassen. Nachteil ist hierbei, wie bei allen Metamodell-basierten Systemen, eine hohe Komplexit�t (\cite{meta}, S.1). \\
Nicht alle Metasysteme sind gleich so komplex, dass sie sich aus mehreren Enterprise-Systemen zusammensetzen. Das Beispiel soll jedoch veranschaulichen, dass Kosten und Nutzen beim Einsatz eines solchen Stils abgew�gt werden m�ssen. Nur, wenn wirklich ein solch hoher Grad an Flexibil�t ben�tigt wird, sollten Metamodelle zum Einsatz kommen. 




\section{Variation Points}
Diese Architekturtaktik besteht darin, Variation Points zu verwenden. 
Dies sind lokale Design-L�sungen, um bestimmte �nderungen an bestimmten Stellen im System zu unterst�tzen. Hierbei m�ssen die Stellen, an denen Variation Points erforderlich sind, identifiziert werden. (\cite{SSA:12}, S.556). \\
Im ersten Abschnitt werden allgemeine Vorgehensweisen pr�sentiert, worauf im zweiten Abschnitt konkrete Design-Patterns folgen, mit denen sich Variability realsieren l�sst. 

\subsection{Vorgehensweisen} 
Ein m�gliches Vorgehen besteht darin, Elemente austauchbar zu gestalten.
Werden die SOLID-Prinzipien konsequent befolgt, ist dies kein Problem. Im Idealfall sind Implementierung und Interface getrennt und die Implementierung h�ngt vom Interface ab. Dann kann die Implementierung des Interfaces durch eine andere Implementierung des Interfaces ausgetauscht werden. Dies entspricht sowohl dem DIP (siehe \ref{dip}) als auch dem LSP (siehe \ref{lsp}). \\
Eine weitere Vorgehensweise besteht darin, Konfigurationen zu verwenden.
Bestimmte Teile des Systemverhaltens lassen sich durch Parametrisierung steuern. Dann lassen sich �nderungen oft alleine durch das Anpassen der Parameter realisieren. . \\
Variation Points k�nnen au�erdem erreicht werden, indem selbstbeschreibende Daten sowie eine generische Verarbeitungsweise gew�hlt werden.
Bei solchen selbstbeschreibenden Datenformaten wie z.B. XML lassen sich die Informationen nutzen, um die Daten generisch zu verarbeiten. \\
Au�erdem ist es von Vorteil, die Verarbeitung des Datenformats von der logischen Verarbeitung getrennt zu halten. Dann l�sst sich das Datenformat wesentlich einfacher �ndern, wie z.B. beim Umstieg von CSV zu XML. \\
Zudem sollten gr��ere Prozesse stets in Teilschritte unterteilt werden. Dies bietet den Vorteil, dass einzelne Schritte austauschbar sind (\cite{SSA:12}, S.556-557).


\subsection{Beispiele}
In diesem Abschnitt werden Design-Patterns vorgestellt, die Variability erm�glichen. 

\subsubsection{Facade}
Dieses Pattern bietet Variability bei der Verwendung von Biblithoken, Frameworks oder einer anderen komplexen Menge an Klassen, indem 
es eine vereinfachte Schnittstelle zu jenen Elementen zur Verf�gung stellt (\cite{pat}, S.210). Diese Schnittstelle enth�lt nur die Methoden, die wirklich ben�tigt werden. Damit ist das System nicht mehr so stark an die externe Bibliothek, das Framework etc. gekoppelt und �nderungen daran, die zu erwarten sind, wirken sich weniger stark aus (\cite{pat}, S.211). Upgrades zu neueren Versionen oder das Austauschen der Software hinter der Schnittstelle stellen mit diesem Design-Pattern kein Problem mehr dar (\cite{pat}, S.214). 

\subsubsection{Template Method}
Dieses Design Pattern bietet Variation Points innerhalb eines Algorithmus. Das Template Method Pattern definiert das Grundger�st eines Algorithmus in der Superklasse und l�sst Subklassen bestimmte Schritte des Algorithmus �berschreiben, ohne dabei die Struktur des Algorithmus zu ver�ndern. Die Verwendung bietet sich dann an, wenn der Algorithmus zwar grundlegend gleich bleibt, aber Details stellenweise angepasst werden m�ssen. 
Beispielsweise ist in einer Data-Mining-Anwendung Variabilit�t bez�glich des Datenformats erforderlich. Der Algorithmus sollte nicht jedes Mal neu geschrieben werden m�ssen, wenn auf ein anderes Datenformat, z.B. von CSF auf PDF, umgestiegen wird.
 Der Algortihmus wird hierzu in einzelne Methoden unterteilt, wobei jede Methode einen Schritt darstellt. Die Abfolge dieser Methoden wird in eine einzige �bergreifende Template Method geschrieben, die entweder abstrakt ist oder eine Default-Implementierung aufweist. Die Subklasse implementiert dann alle abstrakten Schritte und �berschreibt bestimmte Methoden, wenn dies ben�tigt wird (\cite{pat}, S.381-383).



%\subsubsection{Chain of Responsibility}

%\subsubsection{Template Method}
%\subsubsection{Visitor}
%objectifier pattern
%Template and hook
%Template class https://refactoring.guru/design-patterns/template-method
% Dimensional Class Hierarchies
% Bridge https://refactoring.guru/design-patterns/bridge
% Visitor https://refactoring.guru/design-patterns/visitor
% Facet Classifications
 
\subsubsection{Weitere Patterns}
 Weitere Patterns, die Variability erlauben, sind Bridge (\cite{pat}, S.163-177), Chain of Responsibility (\cite{pat}, S.250-267) sowie das Visitor Pattern (\cite{pat}, S.393-408).


\section{Extension Points}
Diese Architekturtaktik besteht darin, Extension Points zu nutzen.
Extension Points sind Schnittstellen f�r Erweiterungen (\cite{extension}, S.1). Sie geben vor, an welchen Stellen des Systems Erweiterungen ankn�pfen sollen und welche Voraussetzung diese Erweiterungen erf�llen m�ssen. \\
Hinter Extension Points steht ein Extension Mechanismus, d.h. die Art und Weise wie die Erweiterung intern vom System unterst�tzt und umgesetzt wird, inklusive Ber�cksichtigung der Software-Umgebung (\cite{extension}, S.1). \\
Bei vielen Standardsoftwares werde Extension Points mitgeliefert, sodass an diese angekn�pft werden kann. So unterst�tzt die J2EE Plattform beispielsweise die Anbindung neuer Datenbanktypen und externer Systeme(\cite{SSA:12}, S.558). Ein anderes Beispiel f�r Standardsoftware, die Extension Points anbietet, stellt Eclipse dar \cite{extension}. Plug-Ins k�nnen an diese Extension Points ankn�pfen (\cite{eclipse}).\\
Die Vorteile bestehen darin, dass den Entwicklern bei der Erweiterung eine Menge Arbeit abgenommen wird. Zudem ist klar, wo erweitert werden muss und wie. Dennoch sollten Extension Points nicht blind genutzt werden, denn schlecht umgesetzte Extension Mechanisms k�nnen Performanzeinbu�en und erh�hte Komplexit�t mit sich bringen (\cite{extension}, S.6). \\
Es kann sinnvoll sein, selbst Extension Points in das System enzubauen, wenn klar ist, dass andere Entwickler in Zukunft das System erweitern werden. \\
Sollte ein Extension Point nicht passen, kann das Adapter Design-Pattern verwendet werden. Es erlaubt Objekten mit inkompatiblen Schnittstellen
 dennoch zusammenzuarbeiten (\cite{pat}, S.150). 



\section{Reliable Change}
Bei dieser Architekturtaktik geht es darum, �nderungen so zuverl�ssig und sicher wie m�glich umzusetzen (\cite{SSA:12}, S.558). Es gibt einige Ma�nahmen, um dies zu erreichen. \\

\subsection{Konfigurationsmanagement}
An erster Stelle ist hier das Konfigurationsmanagement zu nennen. Konfigurationsmangament fasst alle Aktivit�ten zusammen, die zur Verwaltung der Konfigurationen dienen (\cite{SWT:12}, S.253, Z.29-30). Eine Konfiguration ist \glqq die Anordnung eines Computersystems bzw. einer Komponente oder eines Systems, wie sie durch Anzahl, Beschaffenheit und Verbindung seiner Bestandteile definiert ist \grqq (\cite{SWT:12}, S.253, Z.24-26). F�r Konfigurationsmanagement exisiteren zahlreiche Tools, die eingesetzt werden k�nnen. Konfigurationsmanagement schlie�t Versionskontrolle mit ein (\cite{SWT:12}, S.205).

\subsection{Automatisieren}
Vorg�nge wie der Build-Prozess, der Release-Prozess und das Testen sollten automatisiert werden. So werden die Prozesse zuverl�ssig, konsistent und lassen sich wiederholen mit demselben Ergebnis. \\
Automatisiertes Testen sollte auf keinen Fall vernachl�ssigt werden, denn es ist wichtig, sicherzustellen, dass sich mit den �nderungen keine Fehler eingeschlichen haben. Daf�r m�ssen hunderte bis tausende Tests durchgef�hrt werden, was sich manuell nicht mit vertretbarem Aufwand erreichen l�sst (\cite{SSA:12}, S.559). Auch f�r das Automatisierte Testen existieren zahlreiche Tools.
\subsection{Dependency Analysis}
Eine Depency Analysis l�sst sich mit Tools durchf�hren und gibt hilfreichen Aufschluss �ber unentdeckte Abh�ngigkeiten, die sonst eventuell �bersehen worden w�ren. Anhand von Dependency Analysis lassen sich die Auswirkungen von �nderungen absch�tzen (\cite{SSA:12}, S.558).

\subsection{Continuos Integration}
Continuous Integration beschreibt die Vorgehensweise, ge�nderte oder neue Systemelemente so bald wie m�glich in das System zu integrieren und zu testen (\cite{SSA:12}, S.559). \\
Es ist hingegen zu vermeiden, zu warten und alle Elemente auf einmal zu integrieren. Eine solche nicht inkrementelle \textit{big bang} Integration sorgt daf�r, dass alle Probleme gleichzeitig auftreten. Die Lokalisierung und Behebung von Fehlern wird unn�tig erschwert (\cite{SWT:12}, S.60).
\subsection{�nderungen zur�cksetzen}
Nichts bietet so viel Sicherheit wie die Option, eine �nderung jederzeit wieder r�ckg�ngig machen zu k�nnen. Wenn Konfigurationsmanagement beachtet wird, gibt es zwangsl�ufig eine Versionsverwaltung. Wird hierf�r z.B. das Versionsverwaltungssystem Git verwendet, k�nnen �nderungen einfach r�ckg�ngig gemacht werden, z.B. mittels der Befehle \textit{git revert} oder \textit{git reset}.


\chapter{Probleme und Fallen}

%%________________________________________________________________
\section{Statische ADL's}
Um Softwarearchitekturen zu beschreiben wurden viele Architecture Description Languages (ADLs) entwickelt.
In diesen ADLs l�sst sich gut die Gesch�ftslogik abbilden. Viele dieser ADLs erm�glichen es, ein Modell der Komponentenstruktur und des Verhaltens zu erstellen, welche h�ufig als Grundlage f�r die Integrationstests dienen (\cite{SAE}).
\paragraph{Problem}
Viele dieser ADLs bieten keine Repr�sentation f�r die Evolution (\cite{SAE}).
\paragraph{Risiko/Folgen}
Durch die statische Natur der Modelle aus diesen ADLs ist es oft  aufw�ndig die  Softwarearchitekturmodelle bei �nderungen zu pflegen. Dadurch kann es dazu kommen, dass die Software ge�ndert wird, das urspr�nglich Architekturmodell jedoch nicht. Dies verst��t gegen den Concern Erhaltung von Wissen (\ref{Wissen}).
Ein anderes Problem dadurch kann sein, dass das Architekturmodell manuell aktualisiert wird, wodurch jedoch Inkonistenzen eingebaut werden (\cite{SAE}).
\paragraph{Risikoreduzierung}
Eine M�glichkeit dies zu umgehen, ist die Verwendung von ADLs, welche auch �nderungsm�glichkeiten beinhalten. Daf�r gibt es mehrere Ans�tze:
\subparagraph{Explizit Dynamische ADLs}:\\
In diesen ADLs werden zus�tzlich zum Softwarearchitekturmodell noch die dynamischen Aspekte spezifiziert. Jedoch ist es sehr anspruchsvoll alle m�glichen �nderungen explizit zu formulieren.
In diese Kategorie fallen z.B. Wright oder AADL (\cite{SAE}, S. 10).
\subparagraph{ Dynamische ADLs mit Rahmen}:\\
In diesem Ansatz wird ein Rahmen f�r die m�glichen �nderungen definiert.
Die Probleme hierbei sind, dass diese Modelle h�ufig nicht in Verbindung mit den komponentenbasierten Plattformen stehen, sondern eher als eine Art Repository zur Evaluation von expliziten �nderungen zu sehen sind. Das andere Problem besteht darin, dass die Anzahl der erlaubten Architekturen in vielen F�llen unendlich ist und die Modellchecker die Evaluation von unendlich gro�en Architekturfamilien nicht unterst�tzen. Beispiele f�r diese Kategorie w�ren UML2.0, SafArchie, ACL und ArchStudio (\cite{SAE}, S. 12).
\paragraph{Aspektorientierte ADLs}
Hier werden die  Aspect-Oriented Software Development (AOSD) Prinzipien direkt auf Architekturebene innerhalb der ADLs angewandt.
Dies ist durchaus sinnvoll, da die Aspektorientierte Programmierung eine M�glichkeit ist, mit Cross-Cutting-Concerns umzugehen, und in der Softwarearchitektur die Perspektiven auch Cross-Cutting-Concerns  beinhalten.
Als ADLs dieser Kategorie seien Fractal Aspect Component (FAC) und TranSAT zu nennen (\cite{SAE}, S. 12ff).\\
Im Folgenden wird erkl�rt, wie es mit TranSAT m�glich ist, die Architektur um neue Concerns zu erweitern. 


\subsection{TranSAT}
\paragraph{Exkurs Aspekt Orientierte Programmierung}
Aspektorientierte Programmierung (AOP) ist ein Programmierparadigma, welches durch �berlegungen aus dem Prinzip Separation Of Concerns (siehe Abschnitt \ref{seperation}) hervorgegangen ist. Mit klassischen objektorientierten Ans�tzen lassen sich die Gesch�ftslogik gut in einzelne gut trennbare Module aufteilen. Jedoch gibt es neben der Gesch�ftslogik noch weitere Concerns, meist technische wie Logging oder Security, f�r welche dann der Code verteilt �ber die einzelnen (gesch�fts-)logischen Modulen vorhanden ist. Diese Cross-Cutting-Concerns, hier Aspekte genannt, werden dementsprechend auch in eigene Module ausgelagert. Dazu werden sogenannte pointcuts spezifiziert, in welchen Aspekte angewendet werden sollen. Als Beispiel etwa: Logge alle Funktionsaufrufe, deren Name mit set beginnt. Dabei wird der Code der Gesch�ftslogik nicht ver�ndert, sonder �ber eine Weaving-Funktion miteinander vernetzt, also an die in den Pointcuts definierten Stellen eingebunden. In \ref{fig:aop} ist zu sehen, wie sich die Modularit�t aufgrund des Ansatzes verbessert.
Eines der ersten und wohl bekanntesten Frameworks ist AspectJ, welches Java um das AOP Paradigma erweitert (\cite{AOP}).

 \begin{figure}
	\centering
	\includegraphics[width=1\textwidth]{images/aspect.jpg}
	\caption{Cross Cutting Concerns in AspectJ (Abbildung aus \cite{asp})}
	\label{fig:aop}
\end{figure}

\newpage
\paragraph{TranSAT Struktur}
TranSAT (Transform Software Architecture Technologies) ist ein Framework, welches es erm�glicht, bestehende Softwarearchitekturmodelle um neue Concerns zu erweitern. Dabei l�sst TranSAT nur konsistente Ver�nderungen zu.\\
Daf�r besitzt TranSAT  ein Basissoftwarearchitekturmodell bzw. auch Softwarearchitekturspezifikation genannt, welches die bisherige Architektur beschreibt. Dieses kann dann mit den Prinzipien der Aspekt Orientierte Programmierung durch neue Concerns erweitert werden. Typische Concerns sind Persistenz, Security oder Transaktionsmanagement (\cite{SAE}, \cite{TranSAT}).
\subparagraph{Concerns:}
Daf�r m�ssen die Concerns unabh�ngig von dem bestehenden Basissoftwarearchitekturmodell bzw. des Kontextes spezifiziert werden. Dies erm�glicht zudem noch eine leichtere Wiederverwendung der Concerns.\\
H�ufig sind diese neuen Concerns Cross-Cutting-Concerns, d.h. sie ver�ndern mehrere Gesch�ftslogikkomponenten, sowie das Interaktionsverhalten zwischen verschiedene Komponenten.\\
Daf�r m�ssen dann die Integrationsregeln spezifiziert werden.\\
Damit TranSAT aus dem Basissoftwarearchitekturmodel und dem Concern ein neues Softwarearchitekturmodell erstellen kann, werden wie in \ref{fig:TranSAT} zu sehen die Konzepte des Adapters und des Weavers verwendet (\cite{TranSAT}).
\subparagraph{Adapter:}
Im Adapter sind wie in \ref{fig:TranSAT2} zu sehen, die Integrationsregeln spezifiziert. Der Adapter kennt den Kontext nicht, besitzt jedoch wie in \ref{fig:TranSAT2} zu sehen eine Point-Cut-Mask. Die Point-Cut-Mask ist eine Art Vertrag, welcher definiert, wie die Umgebung auszusehen hat, damit der Concern richtig in die Architektur integriert werden kann (\cite{TranSAT}).
\subparagraph{Weaver:}
Der Weaver vergleicht die Informationen aus dem Adapter mit denen des bestehenden Basisarchitekturmodells. Dabei vergleicht er die Point-Cut-Mask des Concerns mit der Point-Cut-Definition im Basisarchitekturmodel und pr�ft ob die Integrationsregeln angewandt werden k�nnen. Dann f�gt er den neuen Concern gem��t der Integrationsregeln an den Stellen, an denen die Point-Cut-Mask und  die Point-Cut-Definition �bereinstimmen, in das Basisarchitekturmodell ein und erzeugt so ein neues Architekturmodell\ref{fig:TranSAT} (\cite{TranSAT}).

 \begin{figure}
	\centering
	\includegraphics[width=1\textwidth]{images/TranSAT.png}
	\caption{TranSat �bersicht (Abbildung aus \cite{TranSAT}, S. 4)}
	\label{fig:TranSAT}
\end{figure}


 \begin{figure}
	\centering
	\includegraphics[width=1\textwidth]{images/TranSAT2.png}
	\caption{TranSat meta-model (Abbildung aus \cite{TranSAT}, S. 5)}
	\label{fig:TranSAT2}
\end{figure}
%\paragraph{TranSat Demo: Bankanwendung}

%\subparagraph{Erweiterung um Atomarit�ts Concern}

\newpage
%%________________________________________________________________
\section{Undokumentierte Architekturen}
In vielen Softwareprojekten gibt es keine Architekturdokumentationen. Auff�llig ist, das viele Open-Source-Projekte die Architektur nicht dokumentieren (\cite{Role}, S. 246).\\
Dies verst��t gegen den Concern Erhaltung von Wissen \ref{Wissen}.

\paragraph{Ursachen}
M�gliche Ursachen k�nnten sein, dass sich keine Gedanken �ber die Architektur gemacht wurden und \glqq einfach drauf los programmiert wird \grqq oder dass, wenn sich Gedanken gemacht wurden, diese nicht dokumentiert werden, da die geplante Architektur im Team als \glqq selbstverst�ndlich\grqq gilt (Quelle: Eigene Erfahrung).
  
\paragraph{Folgen}
Im Laufe der Zeit geht das Wissen �ber die Architektur verloren, falls es �berhaupt vorhanden war. Daraus resultiert dann, dass das Projekt im Laufe der Zeit komplexer wird, ohne dass eine �bersicht �ber die Architektur existiert und die Software somit schlecht zu warten oder erweitern ist.

\paragraph{Folgenreduktion}
Im Laufe der Zeit wurden verschiedene Techniken zum Reverseengineering der Architektur erarbeitet (\cite{Role}).
Beispiele daf�r w�ren: Package View, Bunch View (\cite{otam}), ArchDRH View (\cite{ldrt}) oder ACDC View (\cite{acdc}).\\
Sobald dies gemacht wurde, kann das herausgekommene Architekturmodell dann analysiert, evaluiert und verbessert werden.

%%________________________________________________________________
\section{Nicht Gewartete Architekturen/Architekturmodell Anders Als Implementierung}
Viele Projekte besitzen zwar eine Architekturdokumentation, diese ist jedoch h�ufig nicht vertrauensw�rdig (\cite{Role}, S. 249).
\paragraph{Ursachen}
Viele Ans�tze f�r ADLs entkoppeln den implementierten Code von der Architekturbeschreibung (\cite{Role}, S. 4).

\paragraph{Folgen}
Es kann dazu kommen, dass die Architekturbeschreibung nicht mit dem eigentlichen Code �bereinstimmt und somit inkonsistent ist. Der Code kann dadurch geplante Architektureigenschaften verletzen. Und es wird un�bersichtlich, wenn Code und Architekturbeschreibung nicht �bereinstimmen.
Auch hier liegt ein Versto� gegen den Concern Erhaltung von Wissen (siehe Abschnitt \ref{Wissen}) vor.

\paragraph{Risikoreduktion}
Es ist hilfreich ADLs zu verwenden, welche eng mit dem Code gekoppelt sind. Beispiele daf�r w�ren ArchJava, Fractal oder Sofa (\cite{Role}, S. 4).
Einige bieten die M�glichkeit der Codegenerierung und des �berpr�fens, ob der vorhandene Code gegen das Modell verst��t.

\paragraph{Folgenreduktion}
Sollte es sich eingeschlichen haben, dass das Architekturmodell anders ist als die Implementierung, kann genauso vorgegangen werden, wie wenn keine Dokumentation vorliegt. Evtl. kann es hier sinnvoll sein, das soll-Modell und das ist-Modell zu vergleichen.

%%________________________________________________________________
\section{Priorisierung der Falschen Dimensionen}

\paragraph{Ursachen}
Bei der Festlegung f�r die Unterst�tzung von �nderungsm�glichkeiten kann es vorkommen, dass die Dimensionen st�rker ber�cksichtigt werden, die einem bekannt sind.\\
Es ist auch m�glich, dass bestimmte Dimensionen wichtiger erscheinen, als sie sind, da  lautstarke Stakeholder diese besonders h�ufig erw�hnen.

\paragraph{Folgen}
Die Software wird zu komplex und zu teuer. Au�erdem ist es m�glich, dass, wenn  �nderungen in anderen Dimensionen vonn�ten sein sollten, diese sogar schwieriger werden als wenn die Software mit einer einfacheren Architektur implementiert worden w�re.

\paragraph{Risikoreduktion}
Es sollte nur nach ausreichender Analyse entsprechende Unterst�tzung f�r die �nderungen eingebaut werden, um diese richtig zu priorisieren.\\
(\cite{SSA:12}, S. 560)


%%________________________________________________________________
\section{�nderungen die nie kommen}
Es ist schlichtweg unm�glich, eine Softwarearchitektur mit vertretbaren Kosten und Risiken zu entwickeln, welche alle m�glichen �nderungen unterst�tzt.
\paragraph{Ursachen}
Bei der Entscheidung, welche Arten von �nderungen unterst�tzt werden, wurde falsch entschieden.
\paragraph{Folgen}
Die Unterst�tzung von �nderungen verursacht Overhead im Design, in der Implementierung und h�ufig auch zur Laufzeit. Daher verursacht es unn�tige Kosten, wenn die �nderungen die unterst�tzt werden, nie kommen.
\paragraph{Risikoreduktion}
Die Bereitstellung der Unterst�tzung bestimmter Arten von �nderungen durch die Architektur sollte nur mit ausreichender Sicherheit erfolgen, dass diese auch eintreten.
\paragraph{Folgenreduktion}
Sollte auffallen, dass �nderungen, f�r die schon eine eingebaute Unterst�tzung vorhanden ist, nie kommen, so sollte diese Unterst�tzung als Ballast betrachtet  und dementsprechend entfernt werden. \\
(\cite{SSA:12}, S.561)
%%________________________________________________________________
\section{Auswirkungen auf andere kritische Qualit�tseigenschaften}

\paragraph{Ursachen}
Meistens sind hochflexible Systeme unperformant, besitzen eine hohe Komplexit�t und sind dementsprechend teuer in der Entwicklung.
\paragraph{Folgen/Gefahren}
Daher steht die Fokussierung auf Flexibilit�t h�ufig im direkten Widerspruch zu anderen Qualit�tseigenschaften, wie z.B. die Performanz oder die Benutzbarkeit.\\
Zus�tzlich kommt es vor, dass durch die zu starke Fokussierung auf die Flexibilit�t durch Zeitdruck andere Qualit�tseigenschaften vernachl�ssigt werden.

\paragraph{Risikoreduktion}
Es muss f�r ein Projekt eine passende Balance zwischen den verschiedenen Qualit�tseigenschaften gefunden werden. Daf�r ist ein Prozess der kontinuierlichen Beurteilung der Architektur hilfreich (\cite{SSA:12}, S. 561-562).

%%________________________________________________________________
\section{Zu Starke Abh�ngigkeit von spezifischer Hard- oder Software}


\paragraph{Vorteile}
Fremde Soft-/Hardware zu verwenden kann ggf. preiswerter sein als alles selbst zu entwickeln.
Zudem sind dadurch k�rzere Entwicklungszeiten m�glich.
\paragraph{Nachteile/Gefahren}
Es kann sein, dass die verwendeten Komponenten nicht mehr verf�gbar sind. Bei Hardware k�nnte es sein, dass diese kaputt geht und evtl. nicht mehr hergestellt werden und bei Software z.B. dass Vertr�ge auslaufen.\\
Es kann auch sein, dass irgendwann bestimmte Komponenten theoretisch  durch bessere oder preiswertere Komponenten ausgetauscht werden k�nnten, jedoch die eigene Software speziell auf die eine Komponente angepasst ist und somit diese Komponente faktisch nicht austauschbar ist.
\paragraph{Risikoreduktion}

\begin{enumerate}
	\item Vor der Verwendung von spezifischer Hard- oder Software sollte vorher abgewogen werden, ob die Vorteile �berwiegen.
	\item Sollte entschieden worden sein, dass die Vorteile �berwiegen, gilt es die Roadmaps der Verk�ufer, so wie die anderen Faktoren die das Leben der Komponenten beeinflussen, zu ber�cksichtigen. Also z.B. lieber zus�tzliche Hardware zu kaufen, die als Ausfallersatz dient, sollte verwendete Hardware kaputt gehen.
	\item Die Auswirkungen durch �nderungen der spezifischen Komponenten sollte durch eine entsprechende Abstraktion der  Schnittstellen zu diesen erfolgen.
\end{enumerate} (\cite{SSA:12}, S. 562)
%%________________________________________________________________
\section{Verlorengegangene Entwicklungsumgebung}

\paragraph{Ursachen}
Die Entwicklungs- und Testumgebungen gehen leichter verloren, als die Deploymentumgebung.
Zus�tzlich unterliegen Entwicklungsumgebungen einer von der Deploymentumgebung unabh�ngigen Evolution, da sich Entwicklungs- und Supportpriorit�ten sowie Arbeitspensum  sich �ber Zeit �ndern.

\paragraph{Folgen/Gefahren}
Soll nun ein bestimmter Stand einer Entwicklungs- oder Testumgebungen wiederhergestellt werden, fehlt h�ufig dass Wissen dar�ber, was alles ben�tigt wird und was zu tun ist.

\paragraph{Ben�tigtes Wissen bei der Wiederherstellung}
Um eine entsprechende Entwicklungsumgebung wiederherzustellen, m�ssen folgende Fragen beantwortet werden k�nnen.\\
Wird eine spezielle Version einer Library ben�tigt? Oder funktioniert eine neuere Version auch?\\
Welche Werkzeuge (mit welchen Versionen?) werden f�r den kompletten Build- und Releaseprozess ben�tigt?
Sind spezielle Skripte im Einsatz? Werden daf�r irgendwelche Erweiterungen ben�tigt?\\
Ben�tigen die verwendeten Entwicklungswerkzeuge spezielle Patches? \\
Wird eine bestimmte OS-Version ben�tigt? Oder ein bestimmtes Modell von Hardwarekomponenten? Oder sind diese durch neuere Versionen ersetzbar?\\

\paragraph{Risikoreduktion}
Zun�chst sollte der Name, die Version, der Ursprung und der Grund f�r die Einf�hrung, bei der Einf�hrung externer Elemente in die Entwicklungsumgebung dokumentiert werden. Als informelle Textdatei innerhalb des verwendeten Konfigurationsmanagementsystems sollte ausreichen.\\
Um zu vermeiden, dass die so gesammelten Informationen unvollst�ndig sind, sollte am Ende der Konstruktionsphase die Entwicklungsumgebung woanders nur mit den gesammelten Informationen erneut aufgesetzt und getestet werden. Dabei f�llt auf, ob noch etwas vergessen wurde. Sollte dabei auffallen, dass etwas vergessen wurde, sollte dies mit aufgenommen werden und dann erneut aufgesetzt und getestet werden, so wird erspart, dass die L�cken erst auffallen, wenn das Wissen dar�ber bereits vergessen wurde.\\
Wenn m�glich ist es sinnvoll, Hardware-Virtualisierungstechniken zu verwenden, um die Umgebung zu erhalten.\\
Sollten kritische oder nur schwer zu beschaffende Hardwarekomponenten verwendet werden, ist es au�erdem noch sinnvoll davon Ersatzteile zu besitzen.\\
(\cite{SSA:12}, S. 562-563)

%%________________________________________________________________
\section{Ad	Hoc Release Management}
Das Deployen in eine Testumgebung birgt normalerweise kein Risiko. Sollten dabei Fehler auftreten, so k�nnen diese in Ruhe behoben und dann nochmals deployt werden, ohne dass Benutzer betroffen sind.\\
Wenn jedoch Fehler beim Deployen in eine Produktivumgebung entstehen, k�nnen diese von nervig f�r Benutzer und Admins bis hin zu einer Gef�hrdung der kritischen Operationen der Zielorganisation werden.
Daher wird ein Releasemanagement ben�tigt.

\paragraph{Risikoreduktion}
\label{release}
Der Release-Prozess sollte soweit wie m�glich automatisiert werden. Dies erh�ht die Zuverl�ssigkeit und Wiederholbarkeit. Sobald der Release-Prozess einmal automatisiert wurde, sinkt der Aufwand bei weiteren Releases. Zus�tzlich verringert dies die menschlichen Fehler, die w�hrend des Release-Prozess auftreten.
(\cite{SSA:12}, S. 562-563)



% ...
%--------------------------------------------------------------------------
\backmatter                        		% Anhang
%-------------------------------------------------------------------------
\bibliographystyle{geralpha}			% Literaturverzeichnis
\bibliography{literatur}     			% BibTeX-File literatur.bib
%--------------------------------------------------------------------------
\printindex 							% Index (optional)
%--------------------------------------------------------------------------
\begin{appendix}						% Anh�nge sind i.d.R. optional
%   \include{chapters/Glossar}			% Glossar   
\include{chapters/Selbststaendigkeitserklaerung}
\end{appendix}

\end{document}
