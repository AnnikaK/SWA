\kurzfassung

%% deutsch
\paragraph*{}
Im Rahmen dieser Ausarbeitung wird zun�chst vorgestellt, wie sich �nderungen anhand von Ausma�, Dimension und Zeit einordnen lassen. Danach wird die Evolutionsperspektive auf die Softwarearchitektur erl�utert. Hierbei wird zun�chst gezeigt, wie sich die Evolutionsperspektive auf verschiedenen Viewpoints, wie z.B. den Kontext, den Funktionalen, den Informations, den Concurrency und den Development  Viewpoint anwenden l�sst. Daraufhin werden die Concerns der Evolutionsperspektive erkl�rt.
Im Anschluss werden die Aktivit�ten zur Anwendung der Evolutionsperspektive anhand eines Aktivit�tsdiagramms vorgestellt. Hierbei werden die vier Schritte Anforderungen bestimmen, aktuellen Stand bestimmen, Tradeoffs abw�gen und Architekturtaktiken anwenden n�her erl�utert. AIM42 liefert dazu eine Sammlung bestehender Praktiken, mit deren Hilfe Software systematisch verbessert werden kann. Hier werden die drei Phasen von AIM42, sowie phasen�bergreifende Praktiken  beschrieben.
Nach dem allgemeinen Vorgehen wird eine Auswahl von konkreten Architectural Tactics vorgestellt, die zur Evolution von Software-Architekturen angewandt werden k�nnten. Hierbei werden die etablierten SOLID-Prinzipien im Bezug auf Evolution von Softwarearchitekturen pr�sentiert und erl�utert sowie M�glichkeiten zur lokalen Eingrenzung von �nderungsauswirkungen vorgestellt. Als weitere Taktiken wird gezeigt, wie sich Interfaces erweiterbar gestalten lassen, Variation Points einbauen und Extension Points nutzen lassen. Au�erdem werden Metamodelle behandelt, ein Architekturstil, der auf Evolution ausgelegt ist. Den Abschluss der Architectural Tactics bilden Methoden, �nderungen am Code so sicher wie m�glich umzusetzen. Zu allen Taktiken werden anschauliche Beispiele gezeigt.
Abschlie�end wird auf verschiedene Probleme und Fallen eingegangen. Dabei wird erkl�rt wie diese zustande kommen k�nnen, welche Auswirkungen sie besitzen und wie die Eintrittswahrscheinlichkeit reduzierbar ist.

%% englisch
%\paragraph*{}
%The same in english.
