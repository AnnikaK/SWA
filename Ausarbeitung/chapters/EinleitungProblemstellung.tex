\chapter{Einleitung und Problemstellung}

Begonnen werden soll mit einer Einleitung zum Thema, also Hintergrund und Ziel erl�utert werden.

Weiterhin wird das vorliegende Problem diskutiert: Was ist zu l�sen, warum ist es wichtig, dass man dieses Problem l�st und welche L�sungsans�tze gibt es bereits. Der Bezug auf vorhandene oder eben bisher fehlende L�sungen begr�ndet auch die Intention und Bedeutung dieser Arbeit. Dies k�nnen allgemeine Gesichtspunkte sein: Man liefert einen Beitrag f�r ein generelles Problem oder man hat eine spezielle Systemumgebung oder ein spezielles Produkt (z.B. in einem Unternehmen), woraus sich dieses noch zu l�sende Problem ergibt.

Im weiteren Verlauf wird die Problemstellung konkret dargestellt: Was ist spezifisch zu l�sen? Welche Randbedingungen sind gegeben und was ist die Zielsetzung? Letztere soll das
beschreiben, was man mit dieser Arbeit (mindestens) erreichen m�chte.